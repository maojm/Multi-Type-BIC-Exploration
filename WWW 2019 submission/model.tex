\section{Model and Preliminaries}
\label{sec:model}


\emph{Bayesian exploration} is a game between a principal and $T$ agents who arrive one by one, with agent $t$ arriving in round $t$. Each agent $t$ has a type $\theta_t\in\varTheta$, drawn independently from a fixed distribution, and an action space $\A$.  There is uncertainty, captured by a latent ``state of nature" $\omega\in \varOmega$, henceforth simply the \emph{state}, drawn from a Bayesian prior. The reward $r_t = u(\theta_t,a_t,\omega)$ is determined by the type $\theta_t$, the action $a_t$, and the state $\omega$, for some fixed, deterministic function
$u:\varTheta\times \A \times \varOmega \to [0,1]$.  We assume that the sets $\mA$, $\varTheta$ and $\varOmega$ are finite. We use $\omega_0$ as the random variable for the state, and write $\Pr[\omega]$ for $\Pr[\omega_0=\omega]$. Similarly, we write $\Pr[\theta]$ for $\Pr[\theta_t=\theta]$. {\bf NSI: do we assume $r$ is bounded or anything like that?} An instance of the Bayesian exploration game consists of $T$, the type distribution, the action space, the prior over the state, and the reward function.

A solution to the Bayesian exploration game is a randomized online algorithm $\pi$ termed ``recommendation policy" which, at each round $t$, maps the instance and current history to a message $m_t$.  The history includes the rewards and agent type information obtained in the past rounds and current round up until the recommendation is issued.{\bf NSI: is the following footnote required?}\footnote{Formally, for randomized algorithms, we also assume the history contains the random seed.}  We consider three model variants, depending on whether and when the principal learns the agent's type: the type is revealed immediately after the agent arrives (\emph{public types}), the type is revealed only after the principal issues a recommendation (\emph{reported types}), the type is not revealed (\emph{private types}). Hence the history at round $t$ is $\{(r_1,\theta_1),\ldots,(r_{t-1},\theta_{t-1}),r_t\}$ for public types, $\{(r_1,\theta_1),\ldots,(r_{t-1},\theta_{t-1})\}$ for reported types, and $\{r_1,\ldots,r_{t-1}\}$ for private types.

The goal of each agent $t$ is to choose an action $a_t$ so as to maximize her Bayesian-expected reward given the instance, her type $\theta_t$, and the policy $\pi$. The goal of the principal is to choose a policy $\pi$ that maximizes (Bayesian-expected) total reward, \ie $\E[\sum_{t=1}^T r_t]$.

\iffalse
\emph{Bayesian Exploration} is a game between a principal and $T$ agents who arrive one by one. Each round $t$ proceeds as follows: a new agent $t$ arrives, receives a message $m_t$ from the principal, chooses an action $a_t$ from a fixed action space $\A$, and collects a reward $r_t\in [0,1]$ that is immediately observed by the principal. Each agent has a type $\theta_t\in\varTheta$, drawn independently from a fixed distribution. There is uncertainty, captured by a latent ``state of nature" $\omega\in \varOmega$, henceforth simply the \emph{state}, drawn from a Bayesian prior. The reward $r_t = u(\theta_t,a_t,\omega)$ is determined by the triple $(\theta_t,a_t,\omega)$, for some fixed, deterministic function
    $u:\varTheta\times \A \times \varOmega \to [0,1]$.
The messages $m_t$ are generated according to a randomized online algorithm $\pi$ termed ``recommendation policy".

The type distribution, the Bayesian prior, and the recommendation policy are common knowledge. Each agent $t$ knows her own type $\theta_t$, and observes nothing else except the message $m_t$.  We consider three model variants, depending on whether and when the principal learns the agent's type: the type is revealed immediately after the agent arrives (\emph{public types}), the type is revealed only after the principal issues a recommendation (\emph{reported types}), the type is not revealed (\emph{private types}). Each agent $t$ chooses action $a_t$ so as to maximize her Bayesian-expected utility given what she knows. The goal of the principal is to maximize (Bayesian-expected) total reward, \ie $\E[\sum_{t=1}^T r_t]$.

We assume that the sets $\mA$, $\varTheta$ and $\varOmega$ are finite. We use $\omega_0$ as the random variable for the state, and write $\Pr[\omega]$ for $\Pr[\omega_0=\omega]$. Similarly, we write $\Pr[\theta]$ for $\Pr[\theta_t=\theta]$.
\fi

\xhdr{Bayesian-incentive compatibility.}
For public types, we assume that the message in each round is an action, \ie $m_t \in \A$, and the recommendation policy $\pi$ is {\em Bayesian incentive-compatible} (\emph{BIC}). This is without loss of generality, by a suitable version of Myerson's ``revelation principle".

The BIC condition is stated as follows. Let $\EE_{t-1}$ be the event that the agents have followed principal's recommendations up to (but not including) round $t$. For any type $\theta$ and round $t$, let $\pi^t(\theta)$ be the action recommended by $\pi$ in round $t$. The recommendation policy is BIC if for each round $t$, type $\theta$, and and any two actions $a,a'$ such that $\Pr[\pi^t(\theta) = a|\EE_{t-1}] > 0$ it holds that
\begin{align}\label{eq:model-BIC-actions}
\E_\omega\left[\; u(\theta,a,\omega) - u(\theta,a',\omega) \mid \pi^t(\theta) =a, \EE_{t-1}\;\right] \geq 0.
\end{align}
%(The expectation is over the realized state $\omega$ and the random seed of the algorithm.)

For reported types and private types, we restrict each message to be a \emph{menu}, i.e., a mapping from types to actions, and assume that the recommendation policy satisfies a similar but technically different BIC condition. Again, this is without loss of generality by revelation principle. To state the BIC condition, let $\pi^t$ be the menu recommended in round $t$. The recommendation policy $\pi$ is BIC if for each round $t$, type $\theta$, and any two menus $m,m'$ such that
    $\Pr[\pi^t= m|\EE_{t-1}] > 0$, we have
\begin{align}\label{eq:model-BIC-menus}
\E_\omega\left[\; u(\theta,m(\theta),\omega) - u(\theta,m'(\theta),\omega)
    \mid \pi^t = m, \EE_{t-1}\;\right] \geq 0.
\end{align}
\vspace{-1mm}
\xhdr{Explorability and benchmarks.}
%For these definitions, recommendation policies proceed indefinitely. 
For public types, a type-action pair $(\theta,a)\in \Theta\times \A$ is called \emph{eventually-explorable} in state $\omega$ if there is some BIC recommendation policy that, for $T$ large enough, eventually recommends this action to this agent type with positive probability. Then action $a$ is called \emph{eventually-explorable} for type $\theta$ and state $\omega$. The set of all such actions is denoted $\AExp_{\omega,\theta}$. Likewise, for private types, a menu is called \emph{eventually-explorable} in state $\omega$ if there is some BIC recommendation policy that eventually recommends this menu with positive probability. The set of all such menus is denoted $\MExp_{\omega}$.

Our benchmark is the best eventually-explorable recommendation:
\begin{align*}
\OPTpub &= \sum_{\theta \in \varTheta, \omega\in \varOmega} \Pr[\omega] \cdot \Pr[\theta] \cdot \max_{a \in \AExp_{\omega,\theta}} u(\theta, a, \omega).\\
    &\qquad\text{(for public types: actions)}\\
\OPTpri &= \sum_{\omega\in \varOmega} \Pr[\omega] \cdot\max_{m \in \MExp_{\omega}}\sum_{\theta \in \varTheta} \Pr[\theta] \cdot  u(\theta, m(\theta), \omega)\\
&\qquad\text{(for private types: menus)}.
\end{align*}
We have $\OPTpub \geq \OPTpri$, essentially because any BIC policy for private types can be simulated as a BIC policy for public types. We provide an example (Example \ref{exp:simple}) when $\OPTpub > \OPTpri$.

\OMIT{
Note that, for all settings, no BIC recommendation policy can out-perform the corresponding benchmark.  Our main technical contributions are (computationally efficient) policies that get arbitrarily close to these benchmarks as the number of agents grows.}
