
\asedit{ %%%%%%%%%
\xhdr{Related work.}
Bayesian Exploration with homogenous agents was introduced in \cite{Kremer-JPE14}, and largely resolved: for optimal policy in the case of two actions and deterministic utilities \cite{Kremer-JPE14}, for explorability \cite{ICexplorationGames-ec16}, and for regret minimization and stochastic utilities \cite{ICexploration-ec15}.

Bayesian Exploration with heterogenous agents and public types is studied in \cite{ICexploration-ec15}, under a very strong assumption which ensures explorability of all type-action pairs, and in \cite{ICexplorationGames-ec16}, where a fixed tuple of types arrives in each round and plays a game.  \cite{ICexplorationGames-ec16} is similar to the (public types case of) our paper in that it focuses on explorability. Our approach is simpler and more efficient, because we focus on type-action pairs, whereas \cite{ICexplorationGames-ec16} focus on joint actions. 

A very recent paper \cite{Kempe-colt18} (ours is independent work) studies incentivizing exploration with heterogenous agents and private types, but allows monetary transfers. Assuming that each action is preferred by some agent type, they design an algorithm with a (very) low regret, and conclude that \emph{diversity helps} in their setting.

Several papers study ``incentivizing exploration" in substantially different models:
with a social network \cite{Bahar-ec16};
with time-discounted utilities \cite{Bimpikis-exploration-ms17};
with monetary incentives \cite{Frazier-ec14,Kempe-colt18};  
with a continuous information flow and a continuum of agents \cite{Che-13};
with long-lived agents and ``exploration" separate from payoff generation \cite{Bobby-Glen-ec16,Annie-ec18-traps,Liang-ec18};
with fairness \cite{KKMPRVW17}. Also, seminal papers \cite{Bolton-econometrica99,Keller-econometrica05} study scenarios with 
long-lived, exploring agents and no principal.

Recommendation policies with no explicit exploration, and closely related ``greedy algorithm" in multi-armed bandits, have been studied recently \cite{bastani2017exploiting,Sven-aistats18,kannan2018smoothed,externalities-colt18}.
A common theme is that the greedy algorithm performs well under  substantial assumptions on the diversity of types. Yet, it suffers $\Omega(T)$ regret in the worst case.%
\footnote{This is a well-known folklore result in various settings; \eg see \cite{CompetingBandits-itcs18,Sven-aistats18}.}

\OMIT{\cite{Sven-aistats18} consider a ``full-revelation" recommendation system, and show that (under some substantial assumptions) agent heterogeneity leads to exploration.}

Exploration-exploitation tradeoff received much attention over the past decades, usually under the rubric of ``multi-armed bandits", see books \cite{CesaBL-book,Bubeck-survey12,Gittins-book11}. Absent incentives, Bayesian Exploration with public types is a well-studied problem of ``contextual bandits" (with deterministic rewards and a Bayesian prior). A single round of Bayesian Exploration is a version of the Bayesian Persuasion game \cite{Kamenica-aer11}, where the signal observed by the principal is distinct from the state. Exploration-exploitation problems with incentives issues arise in several other scenarios: dynamic pricing, \eg
    \cite{KleinbergL03,BZ09,BwK-focs13},
dynamic auctions \cite{DynAuctions-survey11}, 
    %\cite{AtheySegal-econometrica13,DynPivot-econometrica10,Kakade-pivot-or13},
advertising auctions
    \cite{MechMAB-ec09,DevanurK09,Transform-ec10-jacm},
human computation
    \cite{RepeatedPA-ec14,Ghosh-itcs13,Krause-www13},
and repeated actions, \eg 
    \cite{Amin-auctions-nips13,Amin-auctions-nips14,Jieming-ec18}.
} %%%%%%%% \asedit

