\documentclass[11pt]{article}
\usepackage{fullpage}
\usepackage{mathrsfs}
\usepackage{float}
% ======    PACKAGES
\usepackage{amsmath, amsfonts, amssymb, amsthm, amsbsy, amscd, bm, bbm}
\usepackage{array}
\usepackage{booktabs}
\usepackage{graphicx}
\usepackage[small,bf]{caption}
\setlength{\captionmargin}{30pt}
\usepackage{subcaption}
\captionsetup[sub]{margin=10pt,font=small}
\usepackage{color}
\usepackage{ifthen}
\usepackage{xspace}
\usepackage{algorithmic,algorithm}
\usepackage[colorlinks,citecolor={black},urlcolor={black},linkcolor={black}]{hyperref}
\usepackage{url}
\usepackage{tocbibind}
\usepackage{enumerate}
\usepackage{mdframed}
\usepackage{comment}


\DeclareMathOperator*{\argmin}{argmin}

% a very useful package for edits and comments, from David Kempe (USC)
\usepackage{color-edits}
%\usepackage[suppress]{color-edits}  % use this to suppress the package
\addauthor{as}{red}      % as for Alex
\addauthor{jm}{blue}     % jm for Jieming
\addauthor{ni}{green}     % ni for Nicole
\addauthor{sw}{magenta}     % sw for Steven
% e.g. for Alex, provides \asedit{}, \ascomment{} and \asdelete{}.


%%%%%%%%%%%%%%%%%%%%%%%%%%%%%%%%%
\newtheorem{theorem}{Theorem}[section]
\newtheorem{corollary}[theorem]{Corollary}
\newtheorem{conjecture}[theorem]{Conjecture}
\newtheorem{proposition}[theorem]{Proposition}
\newtheorem{definition}[theorem]{Definition}
\newtheorem{lemma}[theorem]{Lemma}
\newtheorem{remark}{Remark}[section]
\newtheorem{claim}{Claim}[section]
\newtheorem{example}{Example}[section]


\def\DKL{\textbf{D}_{KL}}
\def\D{\mathbb{D}}
\def\E{\mathbb{E}}

\def\A{\mathcal{A}}
\def\M{\mathcal{M}}
\def\S{\mathcal{S}}
\def\X{\mathcal{X}}
\def\EX{EX}
\def\OPT{OPT}
\def\Ds{*}

\title{Bayesian Exploration:\\ Incentivizing Exploration with Heterogeneous Agents}

\author{
}
\begin{document}
\maketitle

\section{Introduction}

\subsection{Model}
The Bayesian exploration consists of $T$ rounds. The participants are $T$ agents and a principal. Each agent only participates in one round.

The state of nature $\omega$ is drawn from a Bayesian prior distribution over a finite state space $\varOmega$. We use $\Pr[\omega]$ to denote the probability that state $\omega$ is sampled. We use $\Omega$ as the random variable for the state.

In each round, a new agent comes and its type $\theta$ is sampled from a distribution over a finite type space $\varTheta$. The type distribution is independent with the state distribution. We use $\Pr[\theta]$ to denote the probability type $\theta$ is sampled. We use $\Theta$ as the random variable for the type.

For an agent with type $\theta \in \varTheta$, it can choose an action $a$ from action space $\A$ where $\A$ is a finite set. The utility of the agent is a deterministic function of its type and action, and the state of nature: $u(\theta, a, \omega)$. We also assume the utility is bounded in $[0,1]$.

\subsubsection{Type Information}
In terms of principal's knowledge of the agents' types, we consider three different models:
\begin{itemize}
\item \textbf{Public types:} Principal knows the types of agents.
\item \textbf{Private types, communication allowed:} Only the agent knows its own type. The principal asks each agent to report its type.
\item \textbf{Private types, communication not allowed:}  Only the agent knows its own type. Each agent is not allowed to send any message to the principal.
\end{itemize}

%\subsubsection{Bayesian Incentive-Compatible}
%!TEX root = main.tex

\section{Bayesian Exploration with Public Types}
\label{sec:public}
In this section, we show a BIC recommendation policy for public types (as in Theorem \ref{thm:public}). We formally state Theorem \ref{thm:public} in Section \ref{sec:public_bench} and we prove it in Section \ref{sec:public_single} and \ref{sec:public_main}.

\subsection{Explorability and Benchmark}
\label{sec:public_bench}
In this subsection, we define the benchmark and state our main theorem (Theorem \ref{thm:public}).

\begin{definition}
\label{def:public_exp}
An action $a$ of type $\theta$ is eventually-explorable, for a given state $\omega$, if there exists a BIC recommendation policy $\pi$ and some round $t$ such that $\Pr[\pi^t(\theta)= a]> 0$. The set of all such actions for state $\omega$ and type $\theta$ is denoted as $\A_{\omega,\theta}^{exp}$.
\end{definition}

\begin{definition}[Benchmark]
Define benchmark as
\[
\OPT = \sum_{\theta \in \varTheta, \omega\in \varOmega} \Pr[\omega] \cdot \Pr[\theta] \cdot \max_{a \in \A_{\omega,\theta}^{exp}} u(\theta, a, \omega).
\]
\end{definition}

Notice that for a given state $\omega$ and type $\theta$, any BIC recommendation policy can only recommend an action in $\A_{\omega,\theta}^{exp}$. We can simply get the following claim which says that no BIC recommendation policy can get expected per round reward better than the benchmark:
\begin{claim}
Any BIC recommendation policy of $T$ rounds has expected total reward at most $T \cdot\OPT$.
\end{claim}

On the other hand, we construct a BIC recommendation policy that nearly achieves the benchmark. It's proved in the following subsections.
\begin{theorem}
\label{thm:public}
We have a BIC recommendation policy of $T$ rounds with expected total reward at least $\left(T - C \right) \cdot \OPT$ for some constant $C$ which does not depend on $T$.
\end{theorem}

\subsection{Single-round Exploration}
\label{sec:public_single}
%This section is very similar to the EC16's paper, we include this section to make our paper self-contained.

In this subsection, we consider a single round of the Bayesian exploration. As we only consider one round, we fix the agent's type in this round to be $\theta$.

In a single round, the principal receives a random variable $S$ (we call it a signal) which is a function of the history of previous rounds and randomness of the recommendation policy. This function is determined before all rounds start and is known to the principal and all agents. In this single round, the principal computes a (randomized) mapping from the signal to a suggested action. We specify the signal structure $\S$ of signal $S$ by the signal support $\X$ and a joint distribution on $(\X, \varOmega)$.


\begin{definition}
Consider a single-round of Bayesian exploration when the principal receives signal $S$ with signal structure $\S$. An action $a \in \A$ is called signal-explorable, for a given type $\theta$ and a given realized signal $s$, if there exists a single-round BIC recommendation policy $\pi$ such that $\Pr[\pi(s) = a] > 0$. The set of all such actions is denoted as $\EX_s[\S]$. The signal-explorable set is defined as $\EX[\S] = \EX_S[\S]$.
\end{definition}

Note that $\EX_s[\S]$ is a fixed subset and $\EX[\S]$ is a random variable whose realization is determined by the realization of signal $S$.

\xhdr{Information Monotonicity.}
In the following definition, we define a way to compare two signals. Intuitively, the condition $I(S';\Omega|S)= 0$  means if one is given random variable $S$, one can learn no information from $S'$ about $\Omega$. In this case, we show in Lemma \ref{lem:infomono} that the signal-explorable set of $S'$ is contained the signal-explorable set of $S$.
\begin{definition}
Let $S$ and $S'$ be two random variables and $\S$ and $\S'$ be their signal structures. We say random variable $S$ is at least as informative as random variable $S'$ about state $\Omega$ if $I(S' ; \Omega|S) = 0$. Notice $I(S';\Omega|S)$ does not only depend on the signal structures $\S$ and $\S'$. It can also depend on the joint distribution between $S$ and $S'$.
\end{definition}

\begin{lemma}
\label{lem:infomono}
Let $S$ and $S'$ be two random variables and $\S$ and $\S'$ be their signal structures. If $S$ is at least as informative as $S'$ about state $\Omega$ (i.e. $I(S' ; \Omega|S) = 0$), then $\EX_{s'}[\S'] \subseteq \EX_s[\S]$ for all $s' ,s$ such that $\Pr[S= s, S'= s'] > 0$.
\end{lemma}

\begin{proof}
Consider any BIC recommendation policy $\pi'$ for signal structure $\S'$. We construct $\pi$ for signature structure $\S$ by setting $\Pr[\pi(s) = a] = \sum_{s'} \Pr[\pi'(s') = a] \cdot Pr[S' = s'|S = s]$. Notice that $I(S' ; \Omega|S) = 0$ implies $S'$ and $\Omega$ are independent given $S$, i.e $\Pr[S' = s'|S=s] \cdot \Pr[\Omega = \omega|S=s] = \Pr[S'=s', \Omega = \omega|S=s]$ for all $s,s',\omega$. Therefore, for all $s'$ and $\omega$,  
\begin{align*}
 &\sum_s \Pr[S' = s'|S = s] \cdot \Pr[\Omega = \omega, S= s] \\
=& \sum_s   \Pr[S' = s'|S=s] \cdot \Pr[\Omega = \omega|S=s] \cdot \Pr[S=s] \\
= &\sum_s \Pr[S'=s,\Omega =\omega|S=s] \cdot \Pr[S=s] \\
= &\sum_s \Pr[S=s,S'=s',\Omega =\omega] \\
= &\Pr[\Omega =\omega, S'=s].\\
\end{align*}

 Therefore $\pi'$ being BIC implies that $\pi$ is also BIC. More specifically, for any $a,a' \in \A$ and $\theta \in \varTheta$,  
\begin{align*}
&\sum_{\omega,s} \Pr[\Omega = \omega, S = s] \cdot (u(\theta,a', \omega) - u(\theta,a,\omega)) \cdot \Pr[\pi(s) = a] \\
=&\sum_{\omega,s'}\Pr[\Omega = \omega, S' = s'] \cdot (u(\theta,a', \omega) - u(\theta,a,\omega)) \cdot \Pr[\pi(s') = a] \\
\geq & 0.
\end{align*} 

%It's easy to check $\pi$ is BIC. 

Finally, for any $s', s ,a$ such that $Pr[S' = s',S = s] >0 $ and $\Pr[\pi'(s') = a] >0$, we have $\Pr[\pi(s) = a] > 0$. This implies $\EX_{s'}[\S'] \subseteq \EX_s[\S]$.
\end{proof}

\xhdr{Max-Support Policy.}
We can solve the following LP to check whether a particular action $a_0 \in\A$ is signal-explorable given a particular realized signal $s_0\in\X$. In this LP, we represent a policy $\pi$ as a set of numbers
    $x_{a,s} = \Pr[\pi(s)=a]$,
for each action $a\in \A$ and each feasible signal $s\in \X$. 


\begin{figure}[H]
\begin{mdframed}
\vspace{-5mm}
\begin{alignat*}{2}
\textbf{maximize }   & x_{a_0,s_0}\  \\
\textbf{subject to: }
    & \textstyle \sum_{\omega \in \varOmega, s \in \X} \;
    \Pr[\omega] \cdot \Pr[s | \omega] \cdot 
        \left(u(\theta, a, \omega) - u(\theta, a', \omega)\right) \cdot x_{a,s} \geq 0   &\ & \forall a,a' \in \A \\
    & \textstyle \sum_{a\in \A}\; x_{a,s} = 1,  \ &\ & \forall s \in \X \\
    & x_{a,s} \geq 0,  \ &\ & \forall s \in \X, a\in \A
\end{alignat*}
\end{mdframed}
%\caption{LP}
\label{fig:public_lp}
\end{figure}

Since the constraints in this LP characterize any BIC recommendation policy, it follows that action $a_0$ is signal-explorable given realized signal $s_0$ if and only if the LP has a positive solution. If such solution exists, define recommendation policy $\pi = \pi^{a_0,s_0}$ by setting 
    $\Pr[\pi(s) = a] = x_{a,s}$ for all $a\in \A, s\in \X$. 
Then this is a BIC recommendation policy such that 
    $\Pr[\pi(s_0) = a_0] > 0$.

\begin{definition}
Given a signal structure $\S$, a BIC recommendation policy $\pi$ is called  \emph{max-support} if $\forall s \in \X$  and signal-explorable action $a\in \A$ given $s$, $\Pr[\pi(s) = a] > 0$.
\end{definition}


It is easy to see that we obtain max-support recommendation policy by averaging the $\pi^{a,s}$ policies defined above. Specifically, the following policy is BIC and max-support:
\begin{align}\label{eq:pimax}
\pi^{max} = \frac{1}{|\X|} \sum_{s \in \X} \frac{1}{|\EX_s[\S]|} \sum_{a \in \EX_s[\S]} \pi^{a,s}.
\end{align}

%
%We can get a max-support policy $\pi^{max}$ by averaging over BIC recommendation policies for signal explorable actions.
%\begin{claim}
%\label{clm:pimax}
%The following policy is BIC and max-support:
%\[
%\pi^{max} = \frac{1}{|\X|} \sum_{s \in \X} \frac{1}{|\EX_s[\S]|} \sum_{a \in \EX_s[S]} \pi^{a,s}.
%\]
%\end{claim}

\xhdr{Maximal Exploration.}
%\label{sec:public_maxe}
%This section is very similar to the EC16's paper, we include this section to make our paper self-contained.
Let us design a subroutine, called  MaxExplore, which outputs a sequence of $L_{\theta}$ actions for type $\theta$. We are going to assume $L_{\theta} \geq \max_{a,s: \Pr[\pi^{max}(s)=a] \neq 0} \frac{1}{\Pr[\pi^{max}(s)=a]}$.

The goal of this subroutine MaxExplore is to make sure that for any signal-explorable action $a$ (i.e. $\Pr[\pi^{max}(s) = a] > 0$), $a$ shows up at least once in the sequence with probability exactly 1. On the other hand, we want that the action of each specific location in the sequence has marginal distribution same as $\pi^{max}$.

To achieve this goal, the first idea would be to put $C_a = L_{\theta} \cdot \Pr[\pi^{max}(S) = a]$ copies of action $a$ into a sequence of length $L_{\theta}$ and randomly permute the sequence. However, the problems are that $C_a$ might not be an integer and $C_a$ can be smaller than 1. We solve the later problem by picking $L_{\theta}$ large enough such that $C_a \geq 1$ for all $a$ satisfying $\Pr[\pi^{max}(S) =a] > 0$. We solve the former problem by first putting $\lfloor C_a \rfloor$ copies of action $a$ into the sequence and then sampling the rest ($L_\theta - \sum_a \lfloor C_a \rfloor$ actions) according to $p^{Res}$ such that $p^{Res}(a) = \frac{C_a - \lfloor C_a \rfloor}{L_\theta - \sum_a \lfloor C_a \rfloor}$. For details, see the following pseudo-code (Algorithm \ref{alg:public_explore}).
 \begin{algorithm}[H]
    \caption{Subroutine MaxExplore}
    	\label{alg:public_explore}
    \begin{algorithmic}[1]
	\STATE \textbf{Input:} type $\theta$, realized signal $S$ and signal structure $\S$.
	\STATE \textbf{Output:} a list of actions $\alpha$
	\STATE Compute $\pi^{max}$ as per \eqref{eq:pimax}
	%\IF {$l \leq |\A| \cdot |\varTheta|$}
		\STATE Initialize $Res = L_{\theta}$.
		\FOR {each action $ a \in \A$}
							\STATE $C_a \leftarrow  L_{\theta} \cdot \Pr[\pi^{max}(S) = a]$
                     		\STATE Add $\lfloor C_a \rfloor$ copies of action $a$ into list $\alpha$.
			\STATE $Res \leftarrow Res -\lfloor C_a \rfloor $.
			\STATE $p^{Res}(a)\leftarrow  C_a -  \lfloor C_a\rfloor$
		\ENDFOR
		\STATE $p^{Res}(a) \leftarrow p^{Res}(a) / Res$, $\forall a \in \A$.
		\STATE Sample $Res$ many actions from distribution according to $p^{Res}$ independently and add these actions into $\alpha$.
		\STATE Randomly permute the actions in $\alpha$.
	%\ELSE
		%\STATE Add $L_{\theta}$ copies of the best explored action of type $\theta$ according to signal $S$ into action list $\alpha$.
	%\ENDIF
	\RETURN $\alpha$.	
     \end{algorithmic}
\end{algorithm}

Such conversion is exactly the same as \cite{ICexplorationGames-ec16-working}, we have the following claim.

\begin{claim}
\label{clm:maxexplore}
Given type $\theta$ and realized signal $S$, MaxExplore outputs a sequence of $L_{\theta}$ actions. 
Each action in the sequence marginally distributed as $\pi^{max}$. 
For any action $a$ such that $\Pr[\pi^{max} =a] >0$, $a$ shows up in the sequence at least once with probability exactly 1.
MaxExplore runs in time polynomial in $L_{\theta}$, $|\A|$, $|\varOmega|$ and $|\X|$ (size of the support of the signal).
\end{claim}

\subsection{Main Recommendation Policy}
\label{sec:public_main}
In this subsection, we show our main recommendation policy. Algorithm \ref{alg:public_main} is the main procedure of our recommendation policy. It contains two parts: exploration and exploitation. The exploration explores all the eventually-explorable actions and we will give more explanation below. The exploitation part simply advises the agents to choose the best actions among eventually-explorable actions. 

The exploration part proceeds in phases. In a phase, each type $\theta$ gets a sequence of $L_{\theta}$ actions from MaxExplore using data collected before this phase starts. These $L_{\theta}$ actions are suggested to the first $L_{\theta}$ agents of type $\theta$ in the phase. The phase ends when every type $\theta$ has finished $L_{\theta}$ rounds in the current phase. If in a phase some type $\theta$ has already finished $L_{\theta}$ rounds but the phase hasn't ended, our recommended policy simply suggests the best explored action of type $\theta$ to other agents of type $\theta$. And finally after $|\A| \cdot \varTheta|$ phases, our recommendation policy enters the exploitation part.

See the pseudocode (Algorithm \ref{alg:public_main}) for more details. For implementation, in the exploration part, our recommendation policy activates a separate thread (Sub-$\theta$) whenever type $\theta$ shows up in a round and deactivates the thread after that round. We pick $L_{\theta}$ to be at least $\max_{a,s: \Pr[\pi(s)=a] \neq 0} \frac{1}{\Pr[\pi(s)=a]}$ for all $\pi$ that might be chosen as $\pi^{max}$ in Algorithm \ref{alg:public_main} such that $L_{\theta}$ satisfies the guarantee required in Section \ref{sec:public_single}.

 \begin{algorithm}[H]
    \caption{Main procedure for public types }
    	\label{alg:public_main}
    \begin{algorithmic}[1]
    	\STATE Initial signal $S_1 = \S_1 = \perp$.
	\STATE Initial phase count $l = 1$.
	\STATE Initial index of each type $i_{\theta} = 0$ for all $\theta \in \varTheta$.
	\FOR {$t=1$ to $T$}
		\IF {$l \leq |\A|\cdot \varTheta|$}
			\STATE \textbf{Exploration:}
			\STATE Let $\theta_t$ to be the agent type of current round $t$. Run subroutine Sub-$\theta_t$.
			\IF {every type $\theta$ has finished $L_{\theta}$ rounds in the current phase, i.e. $i_{\theta} \geq L_{\theta}$ for all $\theta$}
				\STATE Start a new phase:
				\STATE $l \leftarrow l + 1$.
				\STATE Let $S_l$ be the set of type-action-reward triples in $t$ rounds and $\S_l$ be the signal structure when fixing $\theta_1,...,\theta_t$.
			\ENDIF
		\ELSE
			\STATE \textbf{Exploitation:} 
			\STATE Suggest the best explored action of type $\theta$ to the agent.
		\ENDIF
	\ENDFOR
     \end{algorithmic}
\end{algorithm}

Algorithm \ref{alg:public_sub} describes the thread for type $\theta$. It is activated whenever the agent in the current round has type $\theta$. In a phase, it first suggests $L_{\theta}$ actions computed by MaxExplore. After that it suggests the best explored action to agents. The thread only uses the information collected before the current phase starts.

 \begin{algorithm}[H]
    \caption{Subroutine for type $\theta$: Sub-$\theta$ }
    	\label{alg:public_sub}
    \begin{algorithmic}[1]
	\FOR {each call from Algorithm \ref{alg:public_main}}
		\IF {this is the first call of the current phase}
			\STATE Use the current $S_l$ and $\S_l$ to compute a list of $L_{\theta}$ actions $\alpha_{\theta} \leftarrow $ MaxExplore($\theta, S_l, \S_l$).
			\STATE Initialize the index of type $\theta$: $i_{\theta} \leftarrow 0$.
		\ENDIF
		\STATE $i_{\theta} \leftarrow i_{\theta} + 1$.
		\IF {$i_{\theta} \leq L_{\theta}$}
			\STATE Suggest action $\alpha_{\theta} [i_{\theta}]$ to the agent.
		\ELSE
			\STATE Suggest the best explored action of type $\theta$ to the agent.
		\ENDIF
	\ENDFOR
     \end{algorithmic}
\end{algorithm}


The proof plan is to first give an upper bound on the expected number of rounds of a phase in Lemma \ref{lem:epoch} and then show in Lemma \ref{lem:exp_public} that Algorithm \ref{alg:public_main} explores all the explorable type-action pairs in $|\A| \cdot |\varTheta|$ phases. We use these two lemmas to prove our main theorem (Theorem \ref{thm:public}) in Corollary \ref{cor:public}.

First of all, it's easy to check by Claim \ref{clm:maxexplore} that for each agent, it is Bayesian incentive compatible to follow the recommended action if previous agents all follow the recommended actions. We have the following claim.
\begin{claim}
\label{clm:public_BIC}
Our recommendation policy in this section is BIC.
\end{claim}

\begin{lemma}
\label{lem:epoch}
For any $l>0$, the expected number of rounds of the first $l$ phases is at most $l \cdot \sum_{\theta\in\varTheta} \frac{L_{\theta}}{\Pr[\theta]}$.
\end{lemma}

\begin{proof}
The expected number of rounds of a phase is at most the expected number of rounds in which each type $\theta$ has shown up at least $L_{\theta}$ times. The latter expectation can be easily bounded by $\sum_{\theta\in\varTheta} \frac{L_{\theta}}{\Pr[\theta]}$. Therefore, the expected number of rounds of the first $l$ phases is at most $l \cdot \sum_{\theta\in\varTheta} \frac{L_{\theta}}{\Pr[\theta]}$.
\end{proof}

Notice that in Algorithm \ref{alg:public_main} the partition of phases depends only on realized types $\theta_1,...,\theta_T$. We then have the following claim.
\begin{claim}
Given $\theta_1,...,\theta_T$ as the realized types in $T$ rounds, the partition of phases in Algorithm \ref{alg:public_main} is fixed.
\end{claim}

\begin{lemma}
\label{lem:exp_public}
For any $l>0$ and a fixed sequence of types $\theta_1,...,\theta_T$ as the types encountered by Algorithm \ref{alg:public_main}, assume Algorithm \ref{alg:public_main} has at least $\min(l, |\A|\cdot |\varTheta|)$ phases.
For a given state $\omega$, if an action $a$ of type $\theta$ can be explored by a BIC recommendation policy $\pi$ at round $l$ (i.e. $ \Pr[\pi^l(\theta)= a]> 0$), then such action is guaranteed to be explored by Algorithm \ref{alg:public_main} by the end of phase $\min(l, |\A|\cdot |\varTheta|)$.
\end{lemma}

\begin{proof}
We prove this by induction on $l$ for $l \leq |\A|\cdot |\varTheta|$. Base case $l=1$ is trivial by Claim \ref{clm:maxexplore}. Assuming the lemma is correct for $l-1$, let's prove it's correct for $l$.

Let $S$ be the history of Algorithm \ref{alg:public_main} by the end of phase $l-1$. By the definition of Algorithm 2, $S = S_{l-1} | \theta_1,...,\theta_T$.  Let $S'$ be the history of $\pi$ in the first $l-1$ rounds. More precisely, $S' = R, H_1,...,H_{l-1}$. Here $R$ is the internal randomness of policy $\pi$ and $H_t = (\Theta_t, A_t, u(\Theta_t, A_t, \Omega))$ is the triple of type, action and reward in round $t$ of policy $\pi$.

First of all, we have
\[
I(S'; \Omega| S) = I(R,H_1,...,H_{l-1}; \Omega| S)  = I(R; \Omega| S) + I(H_1,...,H_{l-1}; \Omega|S, R) = I(H_1,...,H_{l-1}; \Omega|S, R).
\]

By the chain rule of mutual information, we have
\[
 I(H_1,...,H_{l-1}; \Omega|S, R) = I(H_1;\Omega|S,R) + I(H_2;\Omega|S, R ,H_1) + \cdots + I(H_{l-1}; \Omega|S,R,H_1,...,H_{l-2}).
\]

For all $t \in [l-1]$, we have
\begin{align*}
&I(H_t; \Omega|S,R,H_1,...,H_{t-1}) \\
=& I(\Theta_t, A_t, u(\Theta_t, A_t, \Omega); \Omega|S,R,H_1,...,H_{t-1}) \\
=& I(\Theta_t ; \Omega|S,R,H_1,...,H_{t-1}) +  I(A_t, u(\Theta_t, A_t, \Omega); \Omega|S,R,H_1,...,H_{t-1},\Theta_t) \\
=& I(A_t, u(\Theta_t, A_t, \Omega); \Omega|S,R,H_1,...,H_{t-1},\Theta_t). \\
\end{align*}
Notice that the suggested action $A_t$ is a deterministic function of randomness of the recommendation policy $R$,  history of previous rounds $H_1,...,H_{t-1}$ and type in the current round $\Theta_t$. Also notice that, by induction hypothesis, $u(\Theta_t, A_t, \Omega)$ is a deterministic function of $|S,R,H_1,...,H_{t-1},\Theta_t, A_t$. Therefore we have
\[
I(H_t; \Omega|S,R,H_1,...,H_{t-1}) = 0, \forall t \in [l-1].
\]
Then we get
\[
I(S'; \Omega | S) = 0.
\]

By Lemma \ref{lem:infomono}, we know that $\EX[\S'] \subseteq \EX[\S]$. For state $\omega$, there exists a signal $s'$ such that $\Pr[S'=s'|\Omega =\omega] >0 $ and $a \in \EX_{s'} [\S']$. Now let $s$ be the realized value of $S$ given $\Omega = \omega$, we know that $\Pr[S'=s'|S=s] >0$, so $a \in \EX_s[\S]$. By Claim \ref{clm:maxexplore}, we know that at least one agent of type $\theta$ in phase $l$ of Algorithm \ref{alg:public_main} will choose action $a$.

Now we consider the case when $l > |\A| \cdot |\varTheta|$. Define Algorithm $E$ to be the variant of Algorithm \ref{alg:public_main} such that it only does exploration (removing the if-condition and exploitation in Algorithm \ref{alg:public_main}). For $l > |\A| \cdot |\varTheta|$, the above induction proof still work for Algorithm $E$, i.e. for a given state $\omega$, if an action $a$ of type $\theta$ can be explored by a BIC recommendation policy $\pi$ at round $l$, then such action is guaranteed to be explored by Algorithm $E$ by the end of phase $l$. Now we are going to argue that Algorithm $E$ won't explore any new action-type pairs after phase $|\A| \cdot |\varTheta|$. Call a phase exploring if in that phase Algorithm $E$ explores at least one new action-type pair. As there are  $ |\A| \cdot |\varTheta|$ type-action pairs, Algorithm $E$ can have at most $ |\A| \cdot |\varTheta|$ exploring phases. On the other hand, once Algorithm $E$ has a phase that is not exploring, because the signal stays the same after that phase, all phases afterwards are not exploring. Therefore Algorithm $E$ does not have any exploring phases after phase $|\A| \cdot |\varTheta|$. For $l > |\A| \cdot |\varTheta|$, the first $|   \A| \cdot |\varTheta|$ phases of Algorithm \ref{alg:public_main} explores the same set of type-action pairs as the first $l$ phases of Algorithm $E$. This concludes the proof.
\end{proof}

\begin{corollary}[Restatement of Theorem \ref{thm:public}]
\label{cor:public}
We have a BIC recommendation policy (Algorithm \ref{alg:public_main}) of $T$ rounds with expected total reward at least $\left(T - C \right) \cdot \OPT$ for constant $C = |\A| \cdot |\varTheta| \cdot \sum_{\theta\in\varTheta} \frac{L_{\theta}}{\Pr[\theta]}$.
\end{corollary}

\begin{proof}
First of all, by Claim \ref{clm:public_BIC}, Algorithm \ref{alg:public_main} is BIC.

By Lemma \ref{lem:exp_public}, for each state $\omega$, Algorithm \ref{alg:public_main} explores all the eventually explorable actions (i.e. $\A_{\omega,\theta}^{exp}$) for each type $\theta$ by the end of $|\A|\cdot |\varTheta|$ phases. After that, for each type $\theta$, Algorithm \ref{alg:public_main} will always recommend action $ \arg\max_{a \in \A_{\omega,\theta}^{exp}} u(\theta, a, \omega)$. Therefore Algorithm \ref{alg:public_main} gets reward $\OPT$ except rounds in first $|\A|\cdot |\varTheta|$ phases. By Lemma \ref{lem:epoch}, we know that the expected number of rounds of the first  $|\A|\cdot |\varTheta|$ phases is at most $C$. Therefore, Algorithm \ref{alg:public_main} has expected total reward at least  $\left(T - C \right) \cdot \OPT$.

\end{proof} 

%!TEX root = main.tex

\subsection{Extension to Reported Types}
\label{sec:private_c}

\newcommand{\pipub}{\pi_{\term{pub}}}

Let us sketch how to extend our ideas for public types to handle the case of reported types. We'd like to simulate the recommendation policy for public types, call it $\pipub$. We simulate it separately for the exploration part and the exploitation part. The exploitation part is fairly easy: we provide a menu that recommends the best explored action for each agent types. In the exploration part, in each round $t$ we guess the agent type to be $\hat{\theta}_t$, with equal probability among all types. The idea is to simulate $\pipub$ only in \emph{lucky rounds} when we guess correctly, \ie $\hat{\theta}_t=\theta_t$. Thus, 
in each round $t$ we simulate the $\ell_t$-th round of $\pipub$, where $\ell_t$ is the number of lucky rounds before round $t$.

In each round $t$ of exploration, we suggest the following menu. For type $\hat{\theta}_t$, we recommend the same action as $\pipub$ would recommend for this type in the $\ell_t$-th round, namely $\pipub^{\ell_t}(\hat{\theta}_t)$.
For any other type, we recommend the action which has the best expected reward given prior knowledge and the action suggested to type $\hat{\theta}$.  We need to  ensure that the suggested actions in the menu are BIC and do not convey extra information to an agent of type $\hat{\theta}$. When we receive the reported type, we can check whether our guess was correct. If so, we input the type-action-reward triple back to $\pipub$. Else, we ignore this round, as if it never happened.

Thus, our recommendation policy eventually explores the same type-action pairs as $\pipub$. The expected number of rounds increases by the factor of $|\varTheta|$. Thus, we have the following theorem.

\begin{theorem}
\label{thm:reported}
Consider Bayesian Exploration with reported types.
There exists a BIC recommendation policy whose expected total reward is at least $\left(T - C \right) \cdot \OPTpub$, 
for some constant $C$ that depends on the problem instance but not on $T$.
This policy explores all type-action pairs that are eventually-explorable for a given state in the case of public types.
\end{theorem}

%$C = |\A| \cdot |\varTheta| \cdot \sum_{\theta\in\varTheta} \frac{L_{\theta}\cdot |\varTheta|}{\Pr[\theta]}$.


%!TEX root = main.tex

\section{Bayesian Exploration with Private Types}
\label{sec:private_nc}

In this section, we show a $\delta$-BIC scheme when types are private and no communication is allowed (as in Theorem \ref{thm:private_nocc}). We formally state Theorem \ref{thm:private_nocc} in Section \ref{sec:private_bench} and we prove it in Section \ref{sec:private_single}, \ref{sec:private_menu}, \ref{sec:private_maxe} and \ref{sec:private_main}.

\subsection{Explorability and Benchmark}
\label{sec:private_bench}
In this subsection, we define the benchmark and state our main theorem (Theorem \ref{thm:private_nocc}). 

\begin{definition}
A menu $m: \varTheta \rightarrow \A$ is a mapping from the type space $\varTheta$ to the action space $\A$. We use $\M$ to denote the set of all menus.
\end{definition}

\begin{claim}
Each single round of a BIC scheme can be considered as a distribution of menus.
\end{claim}

\begin{definition}
\label{def:private_exp}
A menu $m$ is eventually-explorable, for a given state $\omega$, if there exists a BIC recommendation policy $\pi$ and some round $t$ such that $\Pr[\pi^t= m]> 0$. The set of all such menus is denoted as $\M_{\omega}^{exp}$.
\end{definition}

\begin{definition}[Benchmark]
Define benchmark as 
\[
\OPT = \sum_{\omega\in \varOmega} \Pr[\omega] \cdot\max_{m \in \M_{\omega}^{exp}}\sum_{\theta \in \varTheta} \Pr[\theta] \cdot  u(\theta, m(\theta), \omega).
\]
\end{definition}

\begin{claim}
Any BIC recommendation policy of $T$ rounds has expected total reward at most $T \cdot\OPT$.  
\end{claim}

\begin{theorem}
\label{thm:private_nocc}
For any $\delta > 0$, we have a $\delta$-BIC recommendation policy of $T$ rounds with expected total reward at least $\left(T - C\cdot \log(T) \right) \cdot \OPT$ for some constant $C$ which does not depend on $T$. 
\end{theorem}


\subsection{Single-round Exploration}
\label{sec:private_single}

In this subsection, we consider a single round of the Bayesian exploration. 

\begin{definition}
Consider a single-round of Bayesian exploration when the principal has signal $S$ from signal structure $\S$. A menu $m \in \M$ is called signal-explorable, for a given signal $s$, if there exists a single-round $\delta$-BIC recommendation policy $\pi$ such that $\Pr[\pi(s) = m] > 0$. The set of all such actions is denoted as $\EX^{\delta}_s[\S]$. The signal-explorable set is defined as $\EX^{\delta}[\S] = \EX^{\delta}_S[\S]$. We omit $\delta$ in $\EX^{\delta}[\S]$ when $\delta = 0$. 
\end{definition}

\begin{definition}
We say random variable $S$ is $\alpha$-approximately informative as random variable $S'$ about state $\Omega$ if $I(S' ; \Omega|S) = \alpha$. 
\end{definition}

\begin{lemma} [Approximate Information Monotonicity]
\label{lem:ainfomono}
Let $S$ and $S'$ be two random variables and $\S$ and $\S'$ be their signal structures. If $S$ is $(\delta^2/8)$-approximately informative as $S'$ about state $\Omega$ (i.e. $I(S' ; \Omega|S) \leq \delta^2/8$), then $\EX_{s'}[\S'] \subseteq \EX^{\delta}_s[\S]$  for all $s' ,s$ such that $\Pr[S= s, S'= s'] > 0$.
\end{lemma}

\begin{proof}
We have 
\[
\sum_{s} \Pr[S=s] \cdot \DKL\left(S'\Omega|S=s \| (S'|S=s) \times (\Omega|S=s) \right) = I(S' ; \Omega|S) \leq \delta^2/8. 
\]
By Pinsker's inequality, we have
\begin{align*}
       &\sum_{s} \Pr[S = s] \cdot  \sum_{s', \omega} \left| \Pr[S' = s', \Omega = \omega| S= s] - \Pr[S'=s'|S=s] \cdot \Pr[\Omega = \omega|S=s]\right| \\
\leq & \sum_{s} \Pr[S=s] \cdot \sqrt{2 \DKL\left(S'\Omega|S=s \| (S'|S=s) \times (\Omega|S=s) \right)  } \\
\leq &  \sqrt{2 \sum_{s} \Pr[S=s] \cdot  \DKL\left(S'\Omega|S=s \| (S'|S=s) \times (\Omega|S=s) \right) } \\
\leq &\delta /2. \\
\end{align*}

Consider any $\delta$-BIC scheme $\pi'$ for signal structure $\S'$. We construct $\pi$ for signature structure $\S$ by setting $\Pr[\pi(s) = m] = \sum_{s'} \Pr[\pi'(s') = m] \cdot Pr[S' = s'|S = s]$. 

Now we check $\pi$ is BIC. For any $m,m' \in \M$ and $\theta \in \varTheta$,
\begin{align*}
& \sum_{\omega,s} \Pr[\Omega= \omega] \cdot \Pr[S = s | \Omega = \omega] \cdot \left(u(\theta, m(\theta), \omega) - u(\theta, m'(\theta), \omega)\right) \cdot  \Pr[\pi(s) = m] \\
=&\sum_{\omega,s,s'} \Pr[\Omega = \omega, S = s] \cdot \Pr[ S'=s'|S= s] \cdot \Pr[\pi'(s') = m]   \cdot \left(u(\theta, m(\theta), \omega) - u(\theta, m'(\theta), \omega)\right)\\
\geq &\sum_{\omega,s,s'} \Pr[\Omega = \omega, S = s, S'=s'] \cdot \Pr[\pi'(s') = m]   \cdot \left(u(\theta, m(\theta), \omega) - u(\theta, m'(\theta),
 \omega)\right)\\
& -2 \cdot \sum_{\omega,s,s'} | \Pr[\Omega = \omega, S = s] \cdot \Pr[ S'=s'|S= s] -  \Pr[\Omega = \omega, S = s, S'=s']| \\
= &\sum_{\omega,s'} \Pr[\Omega = \omega, S'=s'] \cdot \Pr[\pi'(s') = m]   \cdot \left(u(\theta, m(\theta), \omega) - u(\theta, m'(\theta),
 \omega)\right)\\
& -2 \cdot \sum_{s} \Pr[S = s] \cdot  \sum_{s', \omega} \left| \Pr[S' = s', \Omega = \omega| S= s] - \Pr[S'=s'|S=s] \cdot \Pr[\Omega = \omega|S=s]\right| \\
\geq&  ~\delta - 2 \cdot \frac{\delta}{2}\\
 =& ~0\\
\end{align*}

We also have for any $s', s ,m$ such that $Pr[S' = s',S = s] >0 $ and $\Pr[\pi'(s') = m] >0$, we have $\Pr[\pi(s) = m] > 0$. This implies $\EX^{\delta}_{s'}[\S'] \subseteq \EX_s[\S]$. 
\end{proof}


For a given signal $s \in \X$ and a menu $m \in \M$, we solve the following LP to check if $m$ is $\delta$-signal-exlporable given signal $s$:

\begin{figure}[H]
\begin{mdframed}
\begin{alignat*}{2}
 & \textbf{maximize }    x_{m,s}\  \\
&  \textbf{subject to: }\\
 & \sum_{\omega \in \varOmega, s' \in \X} \Pr[\omega] \cdot \Pr[s' | \omega] \cdot \left(u(\theta, m'(\theta), \omega) - u(\theta, m''(\theta), \omega) + \delta\right) \cdot x_{m',s'} \geq 0  &\ & \forall m',m'' \in \M, \theta \in \varTheta \\
                       & \sum_{m'\in \M} x_{m',s'} = 1,  \ &\ & \forall s' \in \X \\
                       & x_{m',s'} \geq 0,  \ &\ & \forall s' \in \X, m'\in \M \\
\end{alignat*}
\end{mdframed}
%\caption{LP}
\label{fig:nocc_lp}
\end{figure}

As the above LP constraints characterize all BIC recommendation schemes, we have the following claim. 

\begin{claim}
\label{clm:nocc_lp}
For a given signal $s \in \X$, a menu $m \in \M$ is $\delta$-signal-explorable if and only if the above LP has a positive solution. When the LP has a positive solution, define $\pi^{m,s}$ as $\Pr[\pi^{m,s}(s') = m'] = x_{m',s', } \forall m' \in \M, s' \in \X$. Then $\pi^{m,s}$ is a single-round $\delta$-BIC recommendation policy such that $\Pr[\pi^{m,s}(s) = m] > 0$
\end{claim}

\begin{definition}[Max-support policy]
Given a signal structure $\S$, a recommendation policy $\pi$ is called the $\delta$-max-support policy if $\forall s \in \X$  and $\delta$-signal-explorable menu $m\in \M$ given $s$, $\Pr[\pi(s) = m] > 0$. 
\end{definition}

By Claim \ref{clm:nocc_lp}, we have the following claim.
\begin{claim}
\label{clm:pimax_nocc}
The following $\pi^{max}$ is a $\delta$-BIC and $\delta$-max-support policy. Here $\M'$ is the set of menus with positive solutions in the LP mentioned in Claim \ref{clm:nocc_lp}.
\[
\pi^{max} = \frac{1}{|\X|} \sum_{s \in \X} \frac{1}{|\M'|} \sum_{m \in \M'} \pi^{m,s}.
\]
\end{claim}

\begin{comment}
Sometimes the term $\Pr[s'|\omega]$ in the above LP (i.e. the probability of seeing signal $s'$ given state to be $\omega$) is hard to compute. On the other hand, it could be easy to get a approximation of $\Pr[s'|\omega]$ as $p(s',\omega)$ such that $|p(s',w) -\Pr[s'|\omega]| \leq \beta$ for all $s',\omega$. Then we can solve a modified LP using $p(s',\omega)$ instead of $\Pr[s'|\omega]$:

\begin{figure}[H]
\begin{mdframed}
\begin{alignat*}{2}
 & \textbf{maximize }    x_{m,s}\  \\
&  \textbf{subject to: }\\
 & \sum_{\omega \in \varOmega, s' \in \X} \Pr[\omega] \cdot p(s,\omega) \cdot \left(u(\theta, m'(\theta), \omega) - u(\theta, m''(\theta), \omega)\right) \cdot x_{m',s'} \geq -\delta-\beta |\X|   &\ & \forall m',m'' \in \M, \theta \in \varTheta \\
& \sum_{m'\in \M} x_{m',s'} = 1,  \ &\ & \forall s' \in \X \\
& x_{m',s'} \geq 0,  \ &\ & \forall s' \in \X, m'\in \M \\
\end{alignat*}
\end{mdframed}
%\caption{LP}
\label{fig:nocc_lp_a}
\end{figure}

For the modified LP, we have the following claim:
\begin{claim}
\label{clm:nocc_lp_a}
For a given signal $s \in \X$, if a menu $m \in \M$ is $\delta$-signal-explorable, the above LP has a positive solution. When the LP has a positive solution, define $\pi^{m,s}$ as $\Pr[\pi^{m,s}(s') = m'] = x_{m',s', } \forall m' \in \M, s' \in \X$. Then $\pi^{m,s}$ is a single-round $(\delta+2\beta|\X|)$-BIC recommendation policy such that $\Pr[\pi^{m,s}(s) = m] > 0$
\end{claim}

\begin{definition}[Max-support policy]
Given a signal structure $\S$, a recommendation policy $\pi$ is called the $\delta$-max-support policy if $\forall s \in \X$  and $\delta$-signal-explorable menu $m\in \M$ given $s$, $\Pr[\pi(s) = m] > 0$. 
\end{definition}

By Claim \ref{clm:nocc_lp_a}, we have the following claim.
\begin{claim}
\label{clm:pimax_nocc}
The following $\pi^{max}$ is a $(\delta+2\beta|\X|)$-BIC and $\delta$-max-support policy. Here $\M'$ is the set of menus with positive solutions in the LP mentioned in Claim \ref{clm:nocc_lp_a}.
\[
\pi^{max} = \frac{1}{|\X|} \sum_{s \in \X} \frac{1}{|\M'|} \sum_{m \in \M'} \pi^{m,s}.
\]
\end{claim}
\end{comment}

\subsection{Menu Exploration}
\label{sec:private_menu}
Given a menu $m$, a action-reward pair will be revealed to the algorithm after the round. Assuming the agent is following the menu, such action-reward pair is called a sample of the menu $m$. We use $D_m$ to the distribution of the samples. $D_m$ is random variable depending on the state $\Omega$. For a fixed state $\omega$, we use $D_m(\omega)$ to denote the distribution of the samples of menu $m$. 

\begin{lemma}
\label{lem:deltam}
For any $\alpha > 0$, we can compute $\Delta_m$ which is a function of $B_m = O\left(\ln\left(\frac{1}{\gamma}\right)\right)$ samples of menu $m$ such that for any state $\omega$, 
\[
\Pr[\Delta_m \neq D_m(\omega) | \Omega = \omega] \leq \gamma.
\]
\end{lemma}

\begin{proof}
Let $U$ be the union of the support of $D_m(\omega)$ for all $\omega \in \varOmega$. For each $u \in U$ ($u$ is just a sample of the menu), define $q(u,\omega) = \Pr_{v \sim D_m(\omega)}[v = u]$. Let $\delta_m$ be small enough such that for all $\omega, \omega'$ with $D_m(\omega) \neq D_m(\omega')$, there exists $u \in U$, such that $|q(u,\omega) - q(u,\omega')| > \delta_m$. 

Now we compute $\Delta_m$ as following: Take $B_m = \frac{2}{\delta_m^2}\ln\left(\frac{2|U|}{\gamma}\right) $ samples and set $\hat{q}(u)$ as the empirical frequency of seeing $u$. And set $\Delta_m$ to be some $D_m(\omega)$ such that for all $u \in U$, $|q(u,\omega) - \hat{q}(u)| \leq \delta_m / 2$. Notice that if such $\omega$ exists, $\Delta_m$ will be unique. If no $\omega$ satisfies this, just pick $\Delta_m$ to be an arbitrary $D_m(\omega)$. 

Now let's analyze $\Pr[\Delta_m \neq D_m(\omega)]$. Let's fixed the state $\Omega = \omega$. By Chernoff bound, for each $u \in U$, 
\[
\Pr[|q(u,\omega) -\hat{q}(u)| > \delta_m/2] \leq 2\exp\left(-2 \cdot \left(\frac{\delta_m}{2}\right)^2 \cdot B_m\right) \leq \frac{\gamma}{|U|}.
\]
By union bound, with probability at least $1-\gamma$, we have for all$u \in U$, $|q(u,\omega) - \hat{q}(u)| \leq \delta_m / 2$. This implies $\Delta_m = D_m(\omega)$. 
\end{proof}


\subsection{MaxExplore}
\label{sec:private_maxe}
In this subsection, we are going to assume $L \geq \max_{m,s} \frac{B_m}{ \Pr[\pi^{max}(s)=m]}$. 

 \begin{algorithm}[H]
    \caption{Subroutine MaxExplore}
    	\label{alg:nocc_explore}
    \begin{algorithmic}[1]
	\STATE \textbf{Input:} signal $S$, signal structure $\S$.
	\STATE \textbf{Output:} a list of menus $\mu$
	\STATE Compute $\pi^{max}$ as stated in Claim \ref{clm:pimax_nocc}.
	%\IF {$l \leq |\M| $}
		\STATE Initialize $Res = L$.
		\FOR {each menu $ m \in \M$}
                     		\STATE Add $\lfloor L \cdot \Pr[\pi^{max}(S) = m]\rfloor$ copies of menu $m$ into list $\mu$.
			\STATE $Res \leftarrow Res -\lfloor L \cdot \Pr[\pi^{max}(S) = m] \rfloor $.
			\STATE $p^{Res}(a)\leftarrow  L \cdot \Pr[\pi^{max}(S) = m] -  \lfloor L \cdot \Pr[\pi^{max}(S) = m]\rfloor$
		\ENDFOR 
		\STATE $p^{Res}(m) \leftarrow p^{Res}(m) / Res$, $\forall m \in \M$. 
		\STATE Sample $Res$ many actions from distribution according to $p^{Res}$ independently and add these actions into $\mu$. 
		\STATE Randomly permute the actions in $\mu$.
	%\ELSE
		%\STATE Add $L$ copies of the best explored menu according to signal $S$ into action list $\mu$. 
	%\ENDIF
	\RETURN $\mu$.	 
     \end{algorithmic}
\end{algorithm}

\begin{claim}
\label{clm:maxexplore_nocc}
Given signal $S$, MaxExplore is $\delta$-BIC and explores each $\delta$-signal-explorable menu $m$ at least $B_m$ times.  
\end{claim}


\subsection{Main Scheme}
\label{sec:private_main}
In this subsection, we show our main scheme which explores all the eventually-explorable actions and then recommends the agents the best menu given all history. We pick $L$ to be at least $\max_{m,s:\Pr[\pi(s)=m] >0} \frac{B_m}{ \Pr[\pi(s)=m]}$ for all $\pi$ that might be chosen as $\pi^{max}$ by Algorithm \ref{alg:nocc_main}.

 \begin{algorithm}[H]
    \caption{Main procedure for private types }
    	\label{alg:nocc_main}
    \begin{algorithmic}[1]
    	\STATE Initial signal $S_1 = \S_l= \perp$.
	\STATE Initial phase count $l = 1$. 
	\FOR {$t=1$ to $T$}
		\IF {$t \equiv 1 \pmod L$}
			\STATE Start a new phase:
			\STATE For each explored menu $m$ in the previous phase, use $B_m$ samples to compute $\Delta_m$ stated in Lemma \ref{lem:deltam} with $\gamma =\min\left(\frac{\delta^2}{16|\M|\log(|\varOmega|)},\left( \frac{\delta^2}{32|M|}\right)^2, \frac{1}{T|\M|}\right)$. 
			\STATE If there does not exist a state $w$ which is consistent with $\Delta_m$ ($\Delta_m = D_m(\omega)$) for all explored menu $m$, pick an arbitrary state $\omega$ and set $\Delta_m \leftarrow D_m(\omega)$ for all explored menu $m$. This step just make sure the number of signals is bounded by $|\varOmega|$.
			\STATE Set $S_l$ to be the collection of $\Delta_m$'s for all explored menu $m$.
			\IF {$l \leq |\M|$}
				\STATE Use the current $S_l$ and $\S_l$ to compute a list of $L$ actions $\mu \leftarrow $ MaxExplore($S_l, \S_l$).
			\ELSE
				\STATE Set menu list $\mu$ to be $L$ copies of the menu of best action of each type according to all history. 
			\ENDIF
		\ENDIF
		\STATE Suggest menu $\mu [ (t-1) \mod L + 1]$ to the agent.
	\ENDFOR
     \end{algorithmic}
\end{algorithm}


\begin{claim}
\label{clm:nocc_BIC}
Algorithm \ref{alg:nocc_main} is $\delta$-BIC.
\end{claim}


\begin{lemma}
\label{lem:exp_nocc}
For any $l > 0$, assume Algorithm \ref{alg:nocc_main} has at least $\min(l, |\M|)$ phases. 
For a given state $\omega$, if a menu $m$ can be explored by a BIC recommendation policy $\pi$ at round $l$ (i.e. $ \Pr[\pi^l= m]> 0$), then such menu is guaranteed to be explored $B_m$ times by Algorithm \ref{alg:nocc_main} by the end of phase $\min(l, |\M|)$. 
\end{lemma}

\begin{proof}
The proof is similar to Lemma \ref{lem:exp_public}. We prove by induction on $l$ for $l \leq |\M|$. 

%Base case $l=1$ is trivial by Claim \ref{clm:maxexplore}. Assuming the lemma is correct for $l-1$, let's prove it's correct for $l$. 

Let $S$ be the signal of Algorithm \ref{alg:nocc_main} in phase $l$. Let $S'$ be the history of $\pi$ in the first $l-1$ rounds. More precisely, $S' = R, H_1,...,H_{l-1}$. Here $R$ is the internal randomness of scheme $\pi$ and $H_t = (M_t, A_t,u(\Theta_t, M_t(\Theta_t), \Omega))$ is the menu and the action-reward pair in round $t$ of scheme $\pi$. 

Let $\M'$ to be the set of menus explored in the first $l-1$ phases of Algorithm \ref{alg:nocc_main}. By the induction hypothesis, we have $\forall t\in[l-1]$, $M_t \subseteq \M'$. 

First of all, we have
\[
I(S'; \Omega| S) = I(R,H_1,...,H_{l-1}; \Omega| S)  = I(R; \Omega| S) + I(H_1,...,H_{l-1}; \Omega|S, R) = I(H_1,...,H_{l-1}; \Omega|S, R). 
\]

By the chain rule of mutual information, we have
\[
 I(H_1,...,H_{l-1}; \Omega|S, R) = I(H_1;\Omega|S,R) + I(H_2;\Omega|S, R ,H_1) + \cdots + I(H_{l-1}; \Omega|S,R,H_1,...,H_{l-2}). 
\]

For all $t \in [l-1]$, we have
\begin{align*}
&I(H_t; \Omega|S,R,H_1,...,H_{t-1}) \\
=& I(M_t,A_t, u(\Theta_t, M_t(\Theta_t), \Omega); \Omega|S,R,H_1,...,H_{t-1}) \\
=& I(A_t, u(\Theta_t, M_t(\Theta_t), \Omega); \Omega | S,R,H_1,...,H_{t-1}, M_t)\\
\leq& I(D_{M_t}; \Omega|S,R,H_1,...,H_{t-1},M_t). \\
\end{align*}
The second last step comes from the fact that $M_t$ is a deterministic function of $R,H_1,...,H_{t-1}$. The last step comes from the fact that $(A_t,u(\Theta_t, M_t(\Theta_t), \Omega))$ is independent with $\Omega$ given $D_{M_t}$.

Then we have
\begin{align*}
& I(D_{M_t}; \Omega|S,R,H_1,...,H_{t-1},M_t)\\
=& \sum_{m \in \M'} \Pr[M_t = m] \cdot I(D_m;\Omega | S,R,H_1,...,H_{t-1},M_t = m)\\
\leq& \sum_{m \in M'} \Pr[M_t = m] \cdot I(D_m;\Omega| \Delta_m, M_t =m).\\
\leq& \sum_{m \in M'} \Pr[M_t = m] \cdot H(D_m| \Delta_m, M_t =m).
\end{align*}
The last step comes from the fact that $I(D_m; (S\backslash \Delta_m),R,H_1,...,H_{t-1}|\Omega, \Delta_m, M_t =m) = 0$. By Lemma \ref{lem:deltam}, we know that $\Pr[D_m \neq \Delta_m|M_t = m] \leq \gamma$. By Fano's inequality, we have
\[
H(D_m| \Delta_m, M_t =m) \leq H(\gamma) + \gamma \log(|\varOmega| - 1) \leq 2\sqrt{\gamma} + \gamma \log(|\varOmega| - 1)\leq \frac{\delta^2}{16|\M|}+\frac{\delta^2}{16|\M|}  = \frac{\delta^2}{8|\M|}. 
\]

Therefore we have
\[
I(H_t; \Omega|S,R,H_1,...,H_{t-1}) \leq \frac{\delta^2}{8|\M|}, \forall t \in [l-1].
\]
Then we get 
\[
I(S'; \Omega | S) \leq \delta^2/8.
\]

By Lemma \ref{lem:ainfomono}, we know that $\EX_{s'}[\S'] \subseteq \EX^{\delta}_s[\S]$. By Claim \ref{clm:maxexplore_nocc}, we know that phase $l$ will explore menu $m$ at least $B_m$ times.

When $l > |\M|$, the proof follows from the same argument as the last paragraph of the proof of Lemma \ref{lem:exp_public}.
\end{proof}

\begin{corollary}[Restatement of Theorem \ref{thm:private_nocc}]
\label{cor:private_nocc}
For any $\delta > 0$, we have a $\delta$-BIC recommendation policy of $T$ rounds with expected total reward at least $\left(T - C\cdot \log(T) \right) \cdot \OPT$ for some constant $C$ which does not depend on $T$. 
\end{corollary}

\begin{proof}

First of all, by Claim \ref{clm:nocc_BIC}, Algorithm \ref{alg:nocc_main} is $\delta$-BIC. 

By Lemma \ref{lem:exp_nocc}, for each state $\omega$, Algorithm \ref{alg:nocc_main} explores all the eventually explorable menus (i.e. $\M_{\omega}^{exp}$) for by the end of $|\M|$ phases. 

After that, if the algorithm just pick the best explored menu according to all $\Delta_m$'s, by Lemma \ref{lem:deltam} and $\gamma \leq \frac{1}{T|\M|}$, for a fixed state $\omega$, we know that with probability $1- 1/T$, the algorithm chooses menu $\arg\max_{m \in \M_{\omega}^{exp}} \sum_{\theta \in \varTheta} \Pr[\theta] \cdot u(\theta, m(\theta), \omega)$. And since Algorithm \ref{alg:nocc_main} chooses the best action for each type according to all history, it should at least get $(1-1/T) \cdot \max_{m \in \M_{\omega}^{exp}} \sum_{\theta \in \varTheta} \Pr[\theta] \cdot u(\theta, m(\theta), \omega)$ in expectation.

We know that the expected number of rounds of the first  $|\M|$ is $|\M| \cdot L = O(\ln(T))$. Therefore, Algorithm \ref{alg:nocc_main} has expected total reward at least $T \cdot \OPT- T \cdot (1/T) - O(\ln(T)) = T\cdot \OPT - O(\ln(T))$.

\end{proof}



%!TEX root = main.tex

\section{Comparative Statics for Explorability}
\label{sec:statics}

%\newcommand{\pairs}{\mP_\omega}
%\newcommand{\pairsPub}{\pairs^{\term{pub}}}
%\newcommand{\pairsPri}{\pairs^{\term{pri}}}
\newcommand{\pairs}{\AExp_{\omega}}
\newcommand{\pairsPub}{\pairs^{\term{pub}}}
\newcommand{\pairsPri}{\pairs^{\term{pri}}}

\newcommand{\support}{\term{support}}

The eventually-explorable type-action pairs are affected by the model choice and the diversity of types.  All else equal, settings with greater potential exploration have greater total expected reward in both the benchmark and approximation guarantee.  
%
Our first result shows that models with public or reported types can explore (strictly) more actions for each type than models with private types.  Thus more type information (strictly) improves outcomes.
%
Our second result shows that {\bf NSI: add something here.}

\xhdr{Explorability and the model choice.}
Fix an instance of Bayesian exploration. We will show in Section~\ref{sec:reported} that the eventually-explorable set of type-action pairs in a given state of the world is the same for public and reported types. Let $\pairsPub$ be the set of all type-action pairs $\{(\theta,a)|a\in\AExp_{\omega,\theta}\}$ be the eventually-explorable type-action pairs in state $\omega$ with public (equivalently, reported) types.  Similarly, let $\pairsPri$ be the set of all type-action pairs $(\theta,m(\theta))$ that appear in some eventually-explorable menu $m\in \MExp_{\omega}$ in state $\omega$ with private types.
%\footnote{For private types, rewards for all type-action pairs in $\pairsPri$ is an ``upper bound" on the information available to the principal. The principal may know ``less" than that: it does not learn the agent type when the recommended menu maps multiple types to the chosen action.}

It is fairly easy to argue that $\pairsPri \subseteq \pairsPub$. The idea in the proof is that one can simulate any BIC recommendation policy for private types with a BIC recommendation policy for public types; we omit the details. {\bf NSI: might want to include it if there's time.}

\begin{claim}
	Any type-action pair eventually-explorable with private types is also eventually-explorable with public types: $\pairsPri \subseteq \pairsPub$.
\end{claim}


Interestingly, as the following example shows, $\pairsPri$ can in fact be a {\em strict} subset of $\pairsPub$.  

\begin{example}
	\label{exp:simple}
	There are two states, two types and two actions: 
	$\varOmega = \varTheta = \A = \{0,1\}$. 
	States and types are drawn uniformly at random:
	$\Pr[\omega =0] =\Pr[\theta =0] = \tfrac12$. 
	Rewards are defined in the following table:\\
	\begin{table}[H]
		\centering
		\begin{tabular}{|c||c|c|}
			\hline
			&$a=0$&$a=1$\\
			\hline
			\hline
			$\theta = 0$& $u = 3$ & $u =4$\\
			\hline
			$\theta = 1$& $u = 2$ & $u =0$\\
			\hline
		\end{tabular}
		\quad
		\begin{tabular}{|c||c|c|}
			\hline
			&$a=0$&$a=1$\\
			\hline
			\hline
			$\theta = 0$& $u = 2$ & $u =0$\\
			\hline
			$\theta = 1$& $u = 3$ & $u =4$\\
			\hline
		\end{tabular}
		\caption{Rewards $u(\theta,a,\omega)$ when $\omega =0 $ and $\omega = 1$.}
	\end{table}
\end{example}

Action 0 is preferred by both types when there is no information beyond the prior about the state. Thus in the first round, the principal must recommend action $0$ in order for the policy to be BIC.  Hence type-action pairs $\{(0,0),(1,0)\}$ are eventually-explorable in all models.

In the second round, the principal knows the reward of the first-round agent.  When types are public or reported, the reward together with the type is sufficient information for the principal to learn the state.  Moving forward, the principal can now recommend the higher-reward action for each type (either directly or, in the case of reported types, through a menu).  Thus, type-action pair $(0,1)$ is eventually-explorable when $\omega=0$ and, similarly, type-action pair $(1,1)$ is eventually-explorable when $\omega=1$.

For private types, samples from the first-round menu (which, as argued above, must recommend action $0$ for both types) do not convey any information about the state, as they have the same distribution in both states. Therefore, action $1$ is not eventually-explorable, for either type and either state.  Thus:

\begin{claim}
	In Example \ref{exp:simple}, $\pairsPri$ is a strict subset of $\pairsPub$.
\end{claim}

\xhdr{Explorability and diversity of agent types.} {\bf NSI: this section needs editing.}
Let us discuss whether and how the type distribution affects the explorable set. Fix an instance of Bayesian Exploration with type distribution $\DT$. Let us see how the explorable set changes if we change $\DT$ to some other distribution $\DT'$. 

Let $\pairs$ and $\pairs'$ be the corresponding explorable sets, for each state $\omega$. Let $\support(\DT)$ be the support set of distribution $\DT$, \ie the set of all feasible agent types according to $\DT$.

For public types, we show that the explorable set is determined by the support set of $\DT$, and can only increase if the support set increases:

\begin{claim}\label{cl:statics-diversity-public}
Consider Bayesian Exploration with public types. Then:
\begin{OneLiners}
\item[(a)] if $\support(\DT)=\support(\DT')$ then $\pairs=\pairs'$.
\item[(b)] if $\support(\DT)\subset \support(\DT')$ then $\pairs\subseteq \pairs'$.
\end{OneLiners}
\end{claim}

\begin{proof}[Proof Sketch]
We follow the steps of the proof of Lemma \ref{lem:exp_public}. The idea is that the proof carries through even if $\pi$ is a BIC recommendation policy for the less diverse problem instance, \ie an instance with type distribution $\DT'$ such that $\support(\DT')\subset \support(\DT)$. Then for a given state $\omega$, Algorithm \ref{alg:public_main} for the original distribution $\DT$ instance explores all type-action pairs in $\pairs'$.
\end{proof}

The conclusions in Claim~\ref{cl:statics-diversity-public} apply to reported types, too. This is because explorable sets are the same for public and reported types.

For private types, the situation is more complicated. More types can help for some problem instances. For example, if different types have disjoint sets of available actions (more formally: say, disjoint sets of actions with positive rewards) then we are essentially back to the case of reported types, and the conclusions in Claim~\ref{cl:statics-diversity-public} apply. On the other hand, 
we can use Example \ref{exp:simple} to show that more types can hurt explorability when types are private. Recall that in this example, for private types only action 0 can be recommended. Now consider a less diverse instance in which only type 0 appears. After one agent in that type chooses action 0, the state is reviewed to the principal. For example, when the state $\omega = 0$, action $1$ can be recommended to future agents. This shows that,  in this example, explorable set increases when we have fewer types. 

\bibliographystyle{alpha}
\bibliography{references,bib-abbrv,bib-slivkins,bib-bandits,bib-AGT,bib-ML}

\end{document}
