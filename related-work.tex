\subsection{Related work}
\label{sec:related}

\OMIT{ICexploration-ec15,Bimpikis-exploration-ms17}



The problem of Bayesian Exploration was introduced in \cite{Kremer-JPE14}. The special case of homogenous agents has been largely resolved, in terms of the optimal policy for two actions \cite{Kremer-JPE14}, explorability \cite{ICexplorationGames-ec16}, and regret minimization for stochastic utilities \cite{ICexploration-ec15}. \cite{ICexploration-ec15} also consider an extension to public types, under a very strong assumption geared to ensure that all type-action pairs are explorable. \cite{ICexplorationGames-ec16} allows several agents to arrive in each round and play a game. Agents can have different types, but the tuple of types stays the same from one round to another (and is known to the principal). \cite{Bahar-ec16} enrich the model to allow agents to observe recommendations of their ``friends" in a known social network.

Several other papers study related, but technically different models. \cite{Bimpikis-exploration-ms17} consider a basic setting of Bayesian Exploration, but with time-discounted utilities. \cite{Frazier-ec14} allow monetary incentives. \cite{Che-13} posit a continuous information flow and a continuum of agents. \cite{Bobby-Glen-ec16} propose a mechanism to coordinate costly exploration decisions in social learning. \cite{Sven-aistats18} consider a ``full-revelation" recommendation system, and show that (under some substantial assumptions) agent heterogeneity leads to exploration. Scenarios with long-lived, exploring agents and no principal to coordinate them have been studied in \cite{Bolton-econometrica99,Keller-econometrica05} under the name \emph{strategic experimentation}.

Exploration-exploitation tradeoff received much attention over the past decades, usually under the rubric of ``multi-armed bandits", see  \cite{Bubeck-survey12,Gittins-book11} for background. Absent incentives, Bayesian Exploration with public types is a well-studied problem of ``contextual bandits" (with deterministic rewards and a Bayesian prior). A single round of Bayesian Exploration is a version of the Bayesian Persuasion game \cite{Kamenica-aer11}, where the signal observed by the principal is distinct from the state. Exploration-exploitation problems with incentives issues arise in several other scenarios: dynamic pricing
    \cite{KleinbergL03,BZ09,BwK-focs13},
dynamic auctions
    \cite{AtheySegal-econometrica13,DynPivot-econometrica10,Kakade-pivot-or13},
pay-per-click ad auctions
    \cite{MechMAB-ec09,DevanurK09,Transform-ec10-jacm},
and human computation
    \cite{RepeatedPA-ec14,Ghosh-itcs13,Krause-www13}.


