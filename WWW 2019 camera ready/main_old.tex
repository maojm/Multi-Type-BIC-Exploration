\documentclass[11pt]{article}
\usepackage{fullpage}
\usepackage{mathrsfs}
\usepackage{float}
% ======    PACKAGES
\usepackage{slivkins-setup,slivkins-theorems}
%\usepackage{amsmath, amsfonts, amssymb, amsthm, amsbsy, amscd, bm, bbm}
\usepackage{amsbsy, amscd, bm, bbm}
\usepackage{array}
\usepackage{booktabs}
\usepackage{graphicx}
\usepackage[small,bf]{caption}
\setlength{\captionmargin}{30pt}
\usepackage{subcaption}
\captionsetup[sub]{margin=10pt,font=small}
\usepackage{color}
\usepackage{ifthen}
\usepackage{xspace}
\usepackage{algorithmic,algorithm}
\usepackage[colorlinks,citecolor={black},urlcolor={black},linkcolor={black}]{hyperref}
\usepackage{url}
\usepackage{tocbibind}
\usepackage{enumerate}
\usepackage{mdframed}
\usepackage{comment}


\DeclareMathOperator*{\argmin}{argmin}

% a very useful package for edits and comments, from David Kempe (USC)
\usepackage{color-edits}
%\usepackage[suppress]{color-edits}  % use this to suppress the package
\addauthor{as}{red}      % as for Alex
\addauthor{jm}{blue}     % jm for Jieming
\addauthor{ni}{green}     % ni for Nicole
\addauthor{sw}{magenta}     % sw for Steven
% e.g. for Alex, provides \asedit{}, \ascomment{} and \asdelete{}.


%%%%%%%%%%%%%%%%%%%%%%%%%%%%%%%%%%
%\newtheorem{theorem}{Theorem}[section]
%\newtheorem{corollary}[theorem]{Corollary}
%\newtheorem{conjecture}[theorem]{Conjecture}
%\newtheorem{proposition}[theorem]{Proposition}
%\newtheorem{definition}[theorem]{Definition}
%\newtheorem{lemma}[theorem]{Lemma}
%\newtheorem{remark}{Remark}[section]
%\newtheorem{claim}{Claim}[section]
%\newtheorem{example}{Example}[section]

\newcommand{\term}[1]{\ensuremath{\mathtt{#1}}}

\def\DKL{\textbf{D}_{\term{KL}}}
\def\D{\mathbb{D}}
\def\E{\mathbb{E}}

\def\A{\mathcal{A}}
\def\M{\mathcal{M}}
\def\S{\mathcal{S}}
\def\X{\mathcal{X}}
\def\EX{\term{EX}}
\def\OPT{\term{OPT}}
\def\Ds{*}
\def\EE{\mathcal{E}}
\def\varTheta{\bold{\Theta}}
\def\varOmega{\bold{\Omega}}
\title{Bayesian Exploration with Heterogeneous Agents}



\author{
Nicole Immorlica
\and
Jieming Mao
\and
Aleksandrs Slivkins
\and
Zhiwei Steven Wu
}
\begin{document}
\maketitle


\begin{abstract}
It is common in recommendation systems that users both produce and consume information as they make strategic choices under uncertainty. While a social planner would balance “exploration” and “exploitation” using a multi-armed bandit algorithm, users’ incentives may tilt this balance in favor of exploitation. We consider Bayesian Exploration: a simple model in which the recommendation system (the “principal”) controls the information flow to the users (the “agents”) and strives to incentivize exploration via information asymmetry. A single round of this model is a well-known “Bayesian Persuasion game” from (Kamenica and Gentzkow, 2011). We allow heterogeneous users, relaxing a major assumption from prior work that users have the same preferences from one time step to another. The goal is now to learn the best \emph{personalized} recommendations. We consider several versions of the model, depending on whether the user types are public or private and what is the communication protocol between the principal and the users, and design a near-optimal “recommendation policy” for each version. One particular challenge is that it may be impossible to incentivize some of the user types to take some of the actions, no matter what the principal does or how much time she has. We also investigate how the model choice impacts the set of actions that can possibly be “explored” by each type.
\end{abstract}

\section{Introduction}

\subsection{Ideas and Techniques}
\begin{itemize}
\item Global phase+local phase.
\item Reduce from private type, communication allowed to public types.
\item Information theory.
\end{itemize}

\subsection{Model}
The Bayesian exploration consists of $T$ rounds. The participants are $T$ agents and a principal. Each agent only participates in one round.

The state of nature $\omega$ is drawn from a Bayesian prior distribution over a finite state space $\varOmega$. We use $\Pr[\omega]$ to denote the probability that state $\omega$ is sampled. We use $\Omega$ as the random variable for the state.

In each round, a new agent comes and its type $\theta$ is sampled from a distribution over a finite type space $\varTheta$. The type distribution is independent with the state distribution. We use $\Pr[\theta]$ to denote the probability type $\theta$ is sampled. We use $\Theta$ as the random variable for the type.

For an agent with type $\theta \in \varTheta$, it can choose an action $a$ from action space $\A$ where $\A$ is a finite set. The utility of the agent is a deterministic function of its type and action, and the state of nature: $u(\theta, a, \omega)$. We also assume the utility is bounded in $[0,1]$.

\subsubsection{Type Information}
In terms of principal's knowledge of the agents' types, we consider three different models:
\begin{itemize}
\item \textbf{Public types:} Principal knows the types of agents.
\item \textbf{Private types, communication allowed:} Only the agent knows its own type. The principal asks each agent to report its type in the end of the round.
\item \textbf{Private types, communication not allowed:}  Only the agent knows its own type. Each agent is not allowed to send any message to the principal.
\end{itemize}

\subsubsection{Recommendation Policies and Bayesian Incentive Compatible}
The principal commits to an interactive recommendation policy $\pi$. In each round $t$, $\pi$ takes the type as input if types are public. Then $\pi$ sends recommendation to the agent before the agent chooses an action. Then the chosen action and realized reward are inputted into $\pi$. Finally, if we are in the case when types are private and communication is allowed, the reported type is inputted into $\pi$. The principal's goal is to maximize the total sum of rewards in $T$ rounds. 

By revelation principle, we can wlog assume the recommendation in each round is in some specific form. When types are public, we can assume this recommendation is a recommended action. When types are private, we can assume this recommendation is a mapping from types to recommended actions. We call such mapping a menu.

Now we define the incentive compatibility constraint. Let $\EE_{t-1}$ be the event that the agents have followed principal's recommendations up to (but not including) round $t$. 

When types are public, for any type $\theta$ and round $t$, let $\pi^t(\theta)$ be the action recommended by $\pi$ in round $t$. 
\begin{definition}
When types are public, an interactive recommendation policy $\pi$ is Bayesian Incentive Compatible (BIC) if for all rounds $t$ and type $\theta$, we have
\[
\E[ u(\theta,a,\omega) - u(\theta,a',\omega)| \pi^t(\theta) =a, \EE_{t-1}] \geq 0,
\]
where $a,a'$ are any two actions such that $\Pr[\pi^t(\theta) = a|\EE_{t-1}] > 0$. (The probabilities are over the realized state $\omega$ and internal randomness in $\pi$.)
\end{definition}

When types are private, we will have a slightly different definition. Let $\pi^t$ be the menu recommended by $\pi$  in round $t$. Notice that, when communication is allowed, since the type is reported in the end of each round, we assume it's always Bayesian incentive compatible for each agent to report its true type.  
\begin{definition}
When types are private, an interactive recommendation policy $\pi$ is Bayesian Incentive Compatible (BIC) if for all rounds $t$ and type $\theta$, we have
\[
\E[ u(\theta,m(\theta),\omega) - u(\theta,m'(\theta),\omega)| \pi^t = m, \EE_{t-1}] \geq 0,
\]
where $m,m'$ are any two menus such that $\Pr[\pi^t= m|\EE_{t-1}] > 0$. (The probabilities are over the realized state $\omega$ and internal randomness in $\pi$.)
\end{definition}
%!TEX root = main.tex

\section{Public Types}
\label{sec:public}

In this section, we develop our recommendation policy for public types. Throughout, $\OPT = \OPTpub$.

\begin{theorem}
\label{thm:public}
Consider an arbitrary instance of Bayesian Exploration with public types.
There exists a BIC recommendation policy with expected total reward at least $\left(T - C \right) \cdot \OPT$, for some constant $C$ that depends on the problem instance but not on $T$. This policy explores all type-action pairs that are eventually-explorable for a given state.
\end{theorem}

\subsection{A single round of Bayesian Exploration}
\label{sec:public_single}

\xhdr{Signal and explorability.}
We first analyze what actions can be explored by a BIC policy in a single round $t$ of Bayesian Exploration for public types, as a function of the history. Throughout, we suppress $\theta$ and $t$ from our notation.
%and let $\theta$ be the agent's type.
Let $S$ be a random variable equal to the history at round $t$ (referred to as a {\em signal} throughout this section), $s$ be a realization of $S$, and $\S=\DT(\Omega,\mH)$ be the signal structure: \asedit{the joint distribution of $(\omega,S)$}.  Note different policies induce different histories and hence different signal structures.  Thus it will be important to be explicit about the signal structure throughout this section.

%We posit that the principal receives a signal $S$ before the round starts, where $S$ is a random variable.  The \emph{signal structure} of $S$, denoted $\S$, consists of the support set $\X$ and a joint distribution of $(S, \omega_0)$. Here $S$ represents the history of the previous rounds and the internal random seed, and the signal structure represents information about $S$ available to the current agent. However, for this subsection it is more lucid to consider an arbitrary signal structure.
%
%A single-round recommendation policy $\pi$ is simply a randomized mapping from signal realizations to actions. The policy is called \emph{Bayesian-incentive compatible} (\emph{BIC}) for signal $S$ if for each type $\theta$ and any two actions $a,a'$ such that $\Pr[\pi(S) = a] > 0$ it holds that
%\begin{align}\label{eq:model-BIC-single-round}
%\E\left[\; u(\theta,a,\omega) - u(\theta,a',\omega) \mid \pi(S) =a\;\right] \geq 0.
%\end{align}

%\jmcomment{Reviewer 2 asked what this expectation is over. }

\begin{definition}
	Consider a single-round of Bayesian Exploration when the principal receives signal $S$ with signal structure $\S$. An action $a \in \A$ is called {\em signal-explorable for a realized signal $s$} if there exists a BIC recommendation policy $\pi$ such that $\Pr[\pi(s) = a] > 0$. The set of all such actions is denoted as $\EX_s[\S]$. The {\em signal-explorable set}, denoted $\EX[\S]$, is the random subset of actions $\EX_S[\S]$.
\end{definition}

%\jmcomment{Reviewer 2: The notation about $s, S, \mathcal{S}$ are confusing. Some places say ``realized state $s$'' (e.g., in definition 4.2 and several other places) and some places say ``realized state $S$'' (e.g., in Algorithm 1, Claim 4.6 and several other places). }

%Note that $\EX[\S]$ is a random subset of actions whose realization is determined by $s$.

\xhdr{Information-monotonicity.}
We compare the information content of two signals using the notion of conditional mutual information (see Appendix~\ref{app:info-theory} for background). Essentially, we show that a more informative signal leads to the same or larger explorable set.

\begin{definition}
	We say that signal $S$ \emph{is at least as informative} as signal $S'$ if $I(S' ; \omega\mid S) = 0$.
\end{definition}


Intuitively, the condition $I(S';\omega_0|S)= 0$  means if one is given random variable $S$, one can learn no further information from $S'$ about $\omega_0$. Note that this condition depends not only on the signal structures of the two signals, but also on their joint distribution.

\begin{lemma}
	\label{lem:infomono}
	Let $S,S'$ be two signals with signal structures $\S,\S'$. If $S$ is at least as informative as $S'$, then $\EX_{s'}[\S'] \subseteq \EX_s[\S]$ for all $s' ,s$ such that $\Pr[S= s, S'= s'] > 0$.
\end{lemma}

\begin{proof}
Consider any BIC recommendation policy $\pi'$ for signal structure $\S'$. We construct $\pi$ for signal structure $\S$ by setting $\Pr[\pi(s) = a] = \sum_{s'} \Pr[\pi'(s') = a] \cdot Pr[S' = s'\mid S = s]$. Notice that $I(S' ; \omega_0\mid S) = 0$ implies $S'$ and $\omega_0$ are independent given $S$, i.e $\Pr[S' = s'\mid S=s] \cdot \Pr[\omega_0 = \omega\mid S=s] = \Pr[S'=s', \omega_0 = \omega\mid S=s]$ for all $s,s',\omega$. Therefore, for all $s'$ and $\omega$,
\begin{align*}
& \textstyle \sum_s \Pr[S' = s'\mid S = s] \cdot \Pr[\omega_0 = \omega, S= s]\\
&\qquad=\textstyle \sum_s \Pr[S' = s'\mid S=s]
    \cdot \Pr[\omega_0 = \omega\mid S=s] \cdot \Pr[S=s] \\
&\qquad= \textstyle  \sum_s \Pr[S'=s',\omega_0 =\omega\mid S=s] \cdot \Pr[S=s] \\
&\qquad=\textstyle  \sum_s \Pr[S=s,S'=s',\omega_0 =\omega] \\
&\qquad=\Pr[\omega_0 =\omega, S'=s'].
\end{align*}

Therefore $\pi'$ being BIC implies that $\pi$ is also BIC. Indeed, for any $a,a' \in \A$ and $\theta \in \varTheta$, by plugging in the definition of $\pi$,
\begin{align*}
&\textstyle \sum_{\omega,s}\; \Pr[\omega_0 = \omega, S = s] \cdot (u(\theta,a', \omega) - u(\theta,a,\omega)) \cdot \Pr[\pi(s) = a] \\
&\;= \textstyle \sum_{\omega,s'}\;\Pr[\omega_0 = \omega, S' = s'] \cdot (u(\theta,a', \omega) - u(\theta,a,\omega)) \cdot \Pr[\pi'(s') = a]\\
&\;\geq 0.
\end{align*}

%It's easy to check $\pi$ is BIC.

Finally, for any $s', s ,a$ such that $Pr[S' = s',S = s] >0 $ and $\Pr[\pi'(s') = a] >0$, we have $\Pr[\pi(s) = a] > 0$. This implies $\EX_{s'}[\S'] \subseteq \EX_s[\S]$.
\end{proof}

\xhdr{Max-Support Policy.}
We can solve the following LP to check whether a particular action $a_0 \in\A$ is signal-explorable given a particular realized signal $s_0\in\X$. In this LP, we represent a policy $\pi$ as a set of numbers
    $x_{a,s} = \Pr[\pi(s)=a]$,
for each action $a\in \A$ and each feasible signal $s\in \X$.

\begin{figure}[H]
\begin{mdframed}
\vspace{-3mm}
\begin{alignat*}{2}
&\textbf{maximize }    x_{a_0,s_0}\  \\
&\textbf{subject to: }\\
    & \textstyle \sum_{\omega \in \varOmega, s \in \X} \;
    \Pr[\omega] \cdot \Pr[s \mid  \omega] \cdot\\
        &\left(u(\theta, a, \omega) - u(\theta, a', \omega)\right) \cdot x_{a,s} \geq 0   &\ & \forall a,a' \in \A \\
    & \textstyle \sum_{a\in \A}\; x_{a,s} = 1,  \ &\ & \forall s \in \X \\
    & x_{a,s} \geq 0,  \ &\ & \forall s \in \X, a\in \A
\end{alignat*}
\end{mdframed}
%\caption{LP}
\label{fig:public_lp}
\end{figure}

Since the constraints in this LP characterize any BIC recommendation policy, it follows that action $a_0$ is signal-explorable given realized signal $s_0$ if and only if the LP has a positive solution. If such solution exists, define recommendation policy $\pi = \pi^{a_0,s_0}$ by setting
    $\Pr[\pi(s) = a] = x_{a,s}$ for all $a\in \A, s\in \X$.
Then this is a BIC recommendation policy such that
    $\Pr[\pi(s_0) = a_0] > 0$.

\begin{definition}
Given a signal structure $\S$, a BIC recommendation policy $\pi$ is called  \emph{max-support} if $\forall s \in \X$  and signal-explorable action $a\in \A$ given $s$, $\Pr[\pi(s) = a] > 0$.
\end{definition}


It is easy to see that we obtain max-support recommendation policy by averaging the $\pi^{a,s}$ policies defined above. Specifically, the following policy is BIC and max-support:
\begin{align}\label{eq:pimax}
\pi^{\max} = \frac{1}{|\X|} \sum_{s \in \X} \frac{1}{|\EX_s[\S]|} \sum_{a \in \EX_s[\S]} \pi^{a,s}.
\end{align}

%
%We can get a max-support policy $\pi^{\max}$ by averaging over BIC recommendation policies for signal explorable actions.
%\begin{claim}
%\label{clm:pimax}
%The following policy is BIC and max-support:
%\[
%\pi^{\max} = \frac{1}{|\X|} \sum_{s \in \X} \frac{1}{|\EX_s[\S]|} \sum_{a \in \EX_s[S]} \pi^{a,s}.
%\]
%\end{claim}

\xhdr{Maximal Exploration.}
%\label{sec:public_maxe}
%This section is very similar to the EC16's paper, we include this section to make our paper self-contained.
We design a subroutine \term{MaxExplore} which outputs a sequence of actions with two properties: it includes every signal-explorable action at least once, and each action in the sequence marginally distributed as $\pi^{\max}$. The length of this sequence, denoted $L_{\theta}$, should satisfy
\begin{align}\label{eq:public-L}
L_{\theta} \geq \max_{(a,s)\in \A\times \X \text{ with }\quad \Pr[\pi^{\max}(s)=a] \neq 0} \frac{1}{\Pr[\pi^{\max}(s)=a]}.
\end{align}

This step is essentially from \cite{ICexplorationGames-ec16};
%%%%%%% A.S. removed the citation to the ec16 version.
\OMIT{\jmcomment{Reviewer 2 mentioned that we cite your EC paper twice. I know Alex wants to show that this is from your working paper which is not the version for EC.  But it might confuse the reviewer. Not sure if we should do this.}}
%%%%%%%
 we provide the details below for the sake of completeness.
The idea is to put $C_a = L_{\theta} \cdot \Pr[\pi^{\max}(S) = a]$ copies of each action $a$ into a sequence of length $L_{\theta}$ and randomly permute the sequence.
%\jmcomment{Reviewer 2: Equation (5) used $\pi^{max}(s)$, but rightly after $\pi^{max}(S)$ is used. JM: reviewer's confusion about realized vs random values continue.}
 However, $C_a$ might not be an integer, and in particular may be smaller than 1. The latter issue is resolved by making $L_{\theta}$ sufficiently large. For the former issue, we first put $\lfloor C_a \rfloor$ copies of each action $a$ into the sequence, and then sample the remaining
    $L_\theta - \sum_a \lfloor C_a \rfloor$
actions according to distribution
    $\pRes(a) = \frac{C_a - \lfloor C_a \rfloor}{L_\theta - \sum_a \lfloor C_a \rfloor}$.
For details, see Algorithm \ref{alg:public_explore}.
 \begin{algorithm}[H]
    \caption{Subroutine MaxExplore}
    	\label{alg:public_explore}
    \begin{algorithmic}[1]
	\STATE \textbf{Input:} type $\theta$, signal $S$ and signal structure $\S$.
	\STATE \textbf{Output:} a list of actions $\alpha$
	\STATE Compute $\pi^{\max}$ as per \eqref{eq:pimax}
	%\IF {$l \leq |\A| \cdot |\varTheta|$}
		\STATE Initialize $Res = L_{\theta}$.
		\FOR {each action $ a \in \A$}
							\STATE $C_a \leftarrow  L_{\theta} \cdot \Pr[\pi^{\max}(S) = a]$
                     		\STATE Add $\lfloor C_a \rfloor$ copies of action $a$ into list $\alpha$.
			\STATE $Res \leftarrow Res -\lfloor C_a \rfloor $.
			\STATE $\pRes(a)\leftarrow  C_a -  \lfloor C_a\rfloor$
		\ENDFOR
		\STATE $\pRes(a) \leftarrow \pRes(a) / Res$, $\forall a \in \A$.
		\STATE Sample $Res$ many actions from distribution according to $\pRes$ independently and add these actions into $\alpha$.
		\STATE Randomly permute the actions in $\alpha$.
	%\ELSE
		%\STATE Add $L_{\theta}$ copies of the best explored action of type $\theta$ according to signal $S$ into action list $\alpha$.
	%\ENDIF
	\RETURN $\alpha$.	
     \end{algorithmic}
\end{algorithm}

\begin{claim}
\label{clm:maxexplore}
Given type $\theta$ and signal $S$, MaxExplore outputs a sequence of $L_{\theta}$ actions.
Each action in the sequence marginally distributed as $\pi^{\max}$.
For any action $a$ such that $\Pr[\pi^{\max} =a] >0$, $a$ shows up in the sequence at least once with probability exactly 1.
MaxExplore runs in time polynomial in $L_{\theta}$, $|\A|$, $|\varOmega|$ and $|\X|$ (size of the support of the signal).
\end{claim}

%\jmcomment{
%Reviewer 2: In the definition of Maximal Exploration, you mentioned``the marginal distribution at each location is $\pi^max$''. Whose marginal distribution do you mean? Actions’? But $\pi^{max}$ is a randomized map from S to actions, how could it be a marginal distribution of actions? Do you mean $\pi^{max}(S)$?

%JM: Marginal distribution means is the distribution of one action not the distribution of joint actions.
%}
\subsection{Main Recommendation Policy}
\label{sec:public_main}

Algorithm \ref{alg:public_main} is the main procedure of our recommendation policy. It consists of two parts: \emph{exploration}, which explores all the eventually-explorable actions, and \emph{exploitation}, which simply recommends the best explored action for a given type. The exploration part proceeds in phases. In each phase $\ell$, each type $\theta$ gets a sequence of $L_{\theta}$ actions from MaxExplore using the data collected before this phase starts. The phase ends when every agent type $\theta$ has finished $L_{\theta}$ rounds. We pick parameter $L_{\theta}$ large enough so that the condition
\eqref{eq:public-L} is satisfied for all phases $\ell$ and all possible signals $S=S_\ell$. (Note that $L_{\theta}$ is finite because there are only finitely many such signals.)
  After $|\A| \cdot | \varTheta|$ phases, our recommendation policy enters the exploitation part. See Algorithm \ref{alg:public_main} for  details.

 \begin{algorithm}[t]
    \caption{Main procedure for public types }
    	\label{alg:public_main}
    \begin{algorithmic}[1]
    \STATE Initialization: signal $S_1 = \S_1 = \perp$,
             phase count $\ell = 1$, index $i_{\theta} = 0$ for each type $\theta \in \varTheta$.
	\FOR {rounds $t=1$ to $T$}
		\IF {$\ell \leq |\A|\cdot |\varTheta|$}
		 \STATE \COMMENT{Exploration}
		Call thread $\thread(\theta_t)$.
			\IF {every type $\theta$ has finished $L_{\theta}$ rounds in the current phase ($i_{\theta} \geq L_{\theta}$)}
				\STATE Start a new phase: $\ell \leftarrow \ell + 1$.
				\STATE Let $S_\ell$ be the signal for phase $\ell$: \\
                 \TAB the set of all observed type-action-reward triples.
        \STATE Let $\S_\ell$ be the signal structure for $S_\ell$\\
         \TAB given the realized type sequence $(\theta_1,...,\theta_t)$.
			\ENDIF
		\ELSE
			\STATE \COMMENT{Exploitation}
			Recommend the best explored action for agent type $\theta_t$.
		\ENDIF
	\ENDFOR
     \end{algorithmic}
\end{algorithm}

There is a separate thread for each type $\theta$, denoted $\thread(\theta)$,  which is called whenever an agent of this type shows up; see Algorithm \ref{alg:public_sub}. In a given phase $\ell$, it recommends the $L_{\theta}$ actions computed by MaxExplore, then switches to the best explored action. The thread only uses the information collected before the current phase starts: the signal $S_\ell$ and signal structure $\S_\ell$.

 \begin{algorithm}[h]
    \caption{Thread for agent type $\theta$: $\thread(\theta)$ }
    	\label{alg:public_sub}
    \begin{algorithmic}[1]
	%\FOR {each call from Algorithm \ref{alg:public_main}}
		\IF {this is the first call of $\thread(\theta)$ of the current phase}
			\STATE Compute a list of $L_{\theta}$ actions $\alpha_{\theta} \leftarrow $ MaxExplore($\theta, S_\ell, \S_\ell$).
			\STATE Initialize the index of type $\theta$: $i_{\theta} \leftarrow 0$.
		\ENDIF
		\STATE $i_{\theta} \leftarrow i_{\theta} + 1$.
		\IF {$i_{\theta} \leq L_{\theta}$}
			\STATE Recommend action $\alpha_{\theta} [i_{\theta}]$.
		\ELSE
			\STATE Recommend the best explored action of type $\theta$.
		\ENDIF
	%\ENDFOR
     \end{algorithmic}
\end{algorithm}

The BIC property follows easily from Claim \ref{clm:maxexplore}. The key is that Algorithm \ref{alg:public_main} explores all  eventually-explorable type-action pairs.

%\jmcomment{
%Reviewer 2: I did not follow the the statement and proof of Lemma 4.7. Is a phase equivalent to a round? If so, then why within each phase the algorithm interacts with agents for multiple rounds? If not, then why the lemma refers to both phase $l$ and round $l$? Also $\pi$ is the policy at round l, why it has history which is denoted by $S'$? The definition of $H_t$ seems to imply that $\pi$ is played every round…

%JM: A phase is not a round. The answer to the question is that we are comparing our policy with $l$ phases with other poilcy with $l$ rounds.
%}

\OMIT{The performance analysis proceeds as follows. First, we upper-bound the expected number of rounds of a phase (Lemma \ref{lem:epoch}). Then we show, in Lemma \ref{lem:exp_public}, that Algorithm \ref{alg:public_main} explores all  eventually-explorable type-action pairs in $|\A| \cdot |\varTheta|$ phases. We use these two lemmas to prove the main theorem.}

\OMIT{ %%%%%%%%%
\begin{lemma}
\label{lem:epoch}
The expected number of rounds in each phase $\ell$ at most
$ \sum_{\theta\in\varTheta} \frac{L_{\theta}}{\Pr[\theta]}$.
\end{lemma}

\begin{proof}
Phase ends as soon as each type has shown up at least $L_{\theta}$ times. The expected number of rounds by which this happens is at most
$ \sum_{\theta\in\varTheta} \frac{L_{\theta}}{\Pr[\theta]}$.
\end{proof}
} %%%%%%%%

\OMIT{
Notice that in Algorithm \ref{alg:public_main} the partition of phases depends only on realized types $\theta_1,...,\theta_T$.
\begin{claim}
Given the sequence $\theta_1,...,\theta_T$, the partition of phases in Algorithm \ref{alg:public_main} is fixed.
\end{claim}
} %%%%%

\jmedit{The following lemma compares the exploration of Algorithm \ref{alg:public_main} with $l$ phases and some other BIC recommendation policy with $l$ rounds. Notice that a phase in Algorithm \ref{alg:public_main} has many rounds.}
\begin{lemma}
\label{lem:exp_public}
Fix phase $\ell>0$ and the sequence of agent types $\theta_1,...,\theta_T$. Assume Algorithm \ref{alg:public_main} has been running for at least $\min(l, |\A|\cdot |\varTheta|)$ phases.
For a given state $\omega$, if type-action pair $(\theta,a)$ can be explored by some BIC recommendation policy $\pi$ at round $\ell$ with positive probability, then such action is explored by Algorithm \ref{alg:public_main} by the end of phase $\min(l, |\A|\cdot |\varTheta|)$ with probability $1$.
\end{lemma}

\begin{proof}
We prove this by induction on $\ell$ for $\ell \leq |\A|\cdot |\varTheta|$. Base case $\ell=1$ is trivial by Claim \ref{clm:maxexplore}. Assuming the lemma is correct for $\ell-1$, let's prove it's correct for $\ell$.

Let $S= S_l$ be the signal of Algorithm \ref{alg:public_main} by the end of phase $\ell-1$.  Let $S'$ be the history of $\pi$ in the first $\ell-1$ rounds. More precisely,
    $S' = (R, H_1,...,H_{l-1})$,
where $R$ is the internal randomness of policy $\pi$, and
    $H_t = (\Theta_t, A_t, u(\Theta_t, A_t, \omega_0))$
is the type-action-reward triple in round $t$ of policy $\pi$.

The proof plan is as follows. We first show that $I(S';\omega_0|S) =0 $. Informally, this means the information collected in the first $l-1$ phases of Algorithm \ref{alg:public_main} contains all the information $S'$ has about the state $w_0$. After that, we will use the information monotonicity lemma to show that phase $l$ of Algorithm \ref{alg:public_main} explores all the action-type pairs $\pi$ might explore in round $l$.

First of all, we have
\begin{align*}
I(S'; \omega_0\mid  S)
    &= I(R,H_1,...,H_{l-1}; \omega_0\mid  S)\\
   & = I(R; \omega_0\mid  S) + I(H_1,...,H_{l-1}; \omega_0\mid S, R) \\
    &= I(H_1,...,H_{l-1}; \omega_0\mid S, R).
\end{align*}

By the chain rule of mutual information, we have
\begin{align*}
 &I(H_1,...,H_{l-1}; \omega_0\mid S, R) \\
 &\qquad = I(H_1;\omega_0\mid S,R) + \cdots + I(H_{l-1}; \omega_0\mid S,R,H_1,...,H_{l-2}).
\end{align*}

For all $t \in [l-1]$, we have
\begin{align*}
&I(H_t; \omega_0\mid S,R,H_1,...,H_{t-1}) \\
&\qquad= I(\Theta_t, A_t, u(\Theta_t, A_t, \omega_0); \omega_0\mid S,R,H_1,...,H_{t-1}) \\
&\qquad= I(\Theta_t ; \omega_0\mid S,R,H_1,...,H_{t-1})\\
&\qquad\qquad+  I(A_t, u(\Theta_t, A_t, \omega_0); \omega_0\mid S,R,H_1,...,H_{t-1},\Theta_t) \\
&\qquad= I(A_t, u(\Theta_t, A_t, \omega_0); \omega_0\mid S,R,H_1,...,H_{t-1},\Theta_t).
\end{align*}
Notice that the suggested action $A_t$ is a deterministic function of randomness of the recommendation policy $R$,  history of previous rounds $H_1,...,H_{t-1}$ and type in the current round $\Theta_t$. Also notice that, by induction hypothesis, $u(\Theta_t, A_t, \omega_0)$ is a deterministic function of $S,R,H_1,...,H_{t-1},\Theta_t, A_t$. Therefore we have
\[
I(H_t; \omega_0\mid S,R,H_1,...,H_{t-1}) = 0, \qquad \forall t \in [l-1].
\]
Then we get
$ I(S'; \omega_0 \mid  S) = 0.$

By Lemma \ref{lem:infomono}, we know that $\EX[\S'] \subseteq \EX[\S]$. For state $\omega$, there exists a signal $s'$ such that $\Pr[S'=s'\mid \omega_0 =\omega] >0 $ and $a \in \EX_{s'} [\S']$. Now let $s$ be the realized value of $S$ given $\omega_0 = \omega$, we know that $\Pr[S'=s'\mid S=s] >0$, so $a \in \EX_s[\S]$. By Claim \ref{clm:maxexplore}, we know that at least one agent of type $\theta$ in phase $\ell$ of Algorithm \ref{alg:public_main} will choose action $a$.

Now consider the case when $\ell > |\A| \cdot |\varTheta|$. Define \ALG to be the variant of Algorithm \ref{alg:public_main} such that it only does exploration (removing the if-condition and exploitation in Algorithm \ref{alg:public_main}). For $\ell > |\A| \cdot |\varTheta|$, the above induction proof still work for \ALG, i.e. for a given state $\omega$, if an action $a$ of type $\theta$ can be explored by a BIC recommendation policy $\pi$ at round $\ell$, then such action is guaranteed to be explored by \ALG by the end of phase $\ell$. Now we are going to argue that \ALG won't explore any new action-type pairs after phase $|\A| \cdot |\varTheta|$. Call a phase exploring if in that phase \ALG explores at least one new action-type pair. As there are  $ |\A| \cdot |\varTheta|$ type-action pairs, \ALG can have at most $ |\A| \cdot |\varTheta|$ exploring phases. On the other hand, once \ALG has a phase that is not exploring, because the signal stays the same after that phase, all phases afterwards are not exploring. So, \ALG does not have any exploring phases after phase $|\A| \cdot |\varTheta|$. For $\ell > |\A| \cdot |\varTheta|$, the first $|   \A| \cdot |\varTheta|$ phases of Algorithm \ref{alg:public_main} explores the same set of type-action pairs as the first $\ell$ phases of \ALG.
\end{proof}

\begin{proof}[Proof of Theorem \ref{thm:public}]
Algorithm \ref{alg:public_main} is BIC  by Claim \ref{clm:maxexplore}. By Lemma \ref{lem:exp_public}, Algorithm \ref{alg:public_main} explores all the eventually-explorable type-actions pairs after $|\A|\cdot |\varTheta|$ phases.
After that, for each agent type $\theta$, Algorithm \ref{alg:public_main} recommends the best explored action: \\$ \arg\max_{a \in \AExp_{\omega,\theta}} u(\theta, a, \omega)$ with probability exactly 1.%
\footnote{\asedit{This holds with probability is exactly 1, provided that our algorithm finishes $|\A|\cdot |\varTheta|$  phases. If some undesirable low-probability event happens, \eg if all agents seen so far have had the same type, our algorithm would never finish $|\A|\cdot |\varTheta|$ phases.}}

Therefore Algorithm \ref{alg:public_main} gets reward $\OPT$ except rounds in the first $|\A|\cdot |\varTheta|$ phases.  It remains to prove that the expected number of rounds in exploration (i.e. first $|\A|\cdot |\varTheta|$ phases) does not depend on the time horizon $T$. Let $N_\ell$ be the duration of phase $\ell$.
Recall that the phase ends as soon as each type has shown up at least $L_{\theta}$ times. It follows that
$ \E[N_\ell] \leq  \sum_{\theta\in\varTheta} \frac{L_{\theta}}{\Pr[\theta]}$.
So, one can take $C = |\A|\cdot |\varTheta|\cdot \sum_{\theta\in\varTheta} \frac{L_{\theta}}{\Pr[\theta]}$.
\end{proof}
%\jmcomment{
%Reviewer 2: Above Section 3.3, I also did not follow the proof of Theorem 3.1. It claims that "Algorithm 2 always recommends the best explored action" after some phases, why? Is this a high-probability guarantee or 100 percent guarantee? It seems to me that this should have been a high-probability guarantee since there is a strictly positive probability that the agent comes at each round has exactly the same agent type and in this case you can never learn the other agent's best action. If so, what's the algorithm's success probability? If not, why the above is not a concern? At the end, I also did not see why C depends linearly on the expectation of $N_l$ (the duration of phase l). Do we need to guarantee that each type has shown up at least $L_{\theta}$ times FOR SURE, which may never happen if you get unlucky?

%JM: It seems the reviewer does not understand we are proving expected regret bound. The example mentioned by the reviewer will stop us from getting enough phases. But once we have enough phases, it is 100 percent guarantee.
%} 

%!TEX root = main.tex

\section{Bayesian Exploration with Reported Types}
\label{sec:private_c}
In this section, we show a BIC recommendation policy for reported types. The main idea is to simulate the recommendation policy for public types in Section \ref{sec:public} (let's call it $\pi^{pub}$. We describe how we simulate it separately for the exploration part and the exploitation part. 

In the exploration part, our recommendation policy guesses the type with equal probability among all types in each round. The plan is to simulate $\pi^{pub}$ only in rounds when we guess types correctly. In each round, after we guess the type to be $\hat{\theta}$, we suggest the following menu to agents.
\begin{itemize}
\item For type $\hat{\theta}$, we suggest the action that $\pi^{pub}$ would suggest if the agent has type $\hat{\theta}$. 
\item For some other type which is not $\hat{\theta}$, we suggest the action which has the best expected reward given prior knowledge and the action suggested to type $\hat{\theta}$.  We just need to make sure that the suggested actions in the menu are BIC and do not convey extra information to an agent of type $\hat{\theta}$. 
\end{itemize}
In the end of the round, we get the reported type. If our guess is correct, we input the type-action-reward triple back to $\pi^{pub}$. If our guess is wrong, we totally ignore what we learn in this round. By doing this simulation, similarly as $\pi^{pub}$, our recommendation policy explores all actions in $\A^{exp}_{\omega,\theta}$ for any $\theta \in \varTheta$ when the state is $\omega$. Such simulation makes the exploration part to have more rounds as we guess types correctly with probability $\frac{1}{|\varTheta|}$. The expected number of rounds of the exploration part has an extra factor $|\varTheta|$: $|\A| \cdot |\varTheta| \cdot \sum_{\theta\in\varTheta} \frac{L_{\theta}\cdot |\varTheta|}{\Pr[\theta]}$. 

The exploitation part is much easier to simulate as we don't need to know types to do exploitation. We just provide a menu consists of the best explored actions of all types. This will have the same performance as the exploitation part in $\pi^{pub}$.

To conclude, we have the following theorem.

\begin{theorem}
\label{thm:reported}
We have a BIC recommendation policy of $T$ rounds with expected total reward at least $\left(T - C \right) \cdot \OPT$ for some constant $C = |\A| \cdot |\varTheta| \cdot \sum_{\theta\in\varTheta} \frac{L_{\theta}\cdot |\varTheta|}{\Pr[\theta]}$. When the state is $\omega$, it explores all the actions in $\A^{exp}_{\omega,\theta}$ for any type $\theta \in \varTheta$.
\end{theorem}


%!TEX root = main.tex

\section{Bayesian Exploration with Private Types}
\label{sec:private_nc}

Our recommendation policy for private types satisfies a relaxed version of the BIC property, called \emph{$\delta$-BIC}, where the right-hand side in \eqref{eq:model-BIC-menus} is $-\delta$ for some fixed $\delta>0$. We assume a more permissive behavioral model in which agents follow recommendations of such policy.

The main result is as follows. (Throughout this section, $\OPT = \OPTpri$.)


\begin{theorem}
\label{thm:private_nocc}
Consider Bayesian Exploration with private types, and fix $\delta > 0$. There exists a $\delta$-BIC recommendation policy with expected total reward at least $\left(T - C \log T \right) \cdot \OPT$, where $C$ depends on the problem instance but not on time horizon $T$.
\end{theorem}

Our recommendation policy and proofs for private types have a similar structure as the ones for public types. Our recommendation policy proceeds in phases. In each phase, our recommendation policy explores all the menus that can be explored given information collected so far. The crucial step in the proof is to show that the first $l$ phases of our recommendation policy explores all the menus that could be possibly explored by the first $l$ rounds of any BIC recommendation policies.

The new difficulty for private types comes from the fact that we are exploring menus instead of type-actions pairs and we do not learn the reward of a particular type-action pair immediately. For some menus, we could not learn the full information (i.e. distributions of action-reward pairs) about them with finite many rounds of exploration. Therefore we cannot claim that the first $l-1$ phases of our recommendation policy contains all the useful information that could be possibly collected by the first $l-1$ rounds of any BIC recommendation policies. However, we show that this claim is approximately true if we explore each menu enough times in each phase.

We then show that the approximate version of such claim is good enough if we relax the BIC condition a little bit. In particular, we show an approximate information-monotonicity lemma which states that if our recommendation policy collects almost all the useful information that could be possibly collected by the first $l-1$ rounds of any BIC recommendation policies and we relax our recommendation policy from being BIC to being $\delta$-BIC, our recommendation policy can explore all the menus that could be possibly explored by the first $l$ rounds of any BIC recommendation policies.


\subsection{Single-round Exploration}
\label{sec:private_single}

In this subsection, we consider a single round of the Bayesian exploration.

\begin{definition}
Consider a single-round of Bayesian exploration when the principal has signal $S$ from signal structure $\S$. For any $\delta \geq 0$, a menu $m \in \M$ is called $\delta$-signal-explorable, for a given signal $s$, if there exists a single-round $\delta$-BIC recommendation policy $\pi$ such that $\Pr[\pi(s) = m] > 0$. The set of all such menus is denoted as $\EX^{\delta}_s[\S]$. The $\delta$-signal-explorable set is defined as $\EX^{\delta}[\S] = \EX^{\delta}_S[\S]$. We omit $\delta$ in $\EX^{\delta}[\S]$ when $\delta = 0$.
\end{definition}

\xhdr{Approximate Information Monotonicity.}
In the following definition, we define a way to compare two signals approximately.
\begin{definition}
Let $S$ and $S'$ be two random variables. We say random variable $S$ is $\alpha$-approximately informative as random variable $S'$ about state $\omega_0$ if $I(S' ; \omega_0|S) = \alpha$.
\end{definition}

\begin{lemma}
\label{lem:ainfomono}
Let $S$ and $S'$ be two random variables and $\S$ and $\S'$ be their signal structures. If $S$ is $(\delta^2/8)$-approximately informative as $S'$ about state $\omega_0$ (i.e. $I(S' ; \omega_0|S) \leq \delta^2/8$), then $\EX_{s'}[\S'] \subseteq \EX^{\delta}_s[\S]$  for all $s' ,s$ such that $\Pr[S= s, S'= s'] > 0$.
\end{lemma}

\begin{proof}
We have
\[
\sum_{s} \Pr[S=s] \cdot \DKL\left(S'\omega_0|S=s \| (S'|S=s) \times (\omega_0|S=s) \right) = I(S' ; \omega_0|S) \leq \delta^2/8.
\]
By Pinsker's inequality, we have
\begin{align*}
       &\sum_{s} \Pr[S = s] \cdot  \sum_{s', \omega} \left| \Pr[S' = s', \omega_0 = \omega| S= s] - \Pr[S'=s'|S=s] \cdot \Pr[\omega_0 = \omega|S=s]\right| \\
\leq & \sum_{s} \Pr[S=s] \cdot \sqrt{2 \DKL\left(S'\omega_0|S=s \| (S'|S=s) \times (\omega_0|S=s) \right)  } \\
\leq &  \sqrt{2 \sum_{s} \Pr[S=s] \cdot  \DKL\left(S'\omega_0|S=s \| (S'|S=s) \times (\omega_0|S=s) \right) } \\
\leq &\delta /2. \\
\end{align*}

Consider any BIC recommendation policy $\pi'$ for signal structure $\S'$. We construct $\pi$ for signature structure $\S$ by setting $\Pr[\pi(s) = m] = \sum_{s'} \Pr[\pi'(s') = m] \cdot Pr[S' = s'|S = s]$.

Now we check $\pi$ is $\delta$-BIC. For any $m,m' \in \M$ and $\theta \in \varTheta$,
\begin{align*}
& \sum_{\omega,s} \Pr[\omega_0= \omega] \cdot \Pr[S = s | \omega_0 = \omega] \cdot \left(u(\theta, m(\theta), \omega) - u(\theta, m'(\theta), \omega)\right) \cdot  \Pr[\pi(s) = m] \\
=&\sum_{\omega,s,s'} \Pr[\omega_0 = \omega, S = s] \cdot \Pr[ S'=s'|S= s] \cdot \Pr[\pi'(s') = m]   \cdot \left(u(\theta, m(\theta), \omega) - u(\theta, m'(\theta), \omega)\right)\\
\geq &\sum_{\omega,s,s'} \Pr[\omega_0 = \omega, S = s, S'=s'] \cdot \Pr[\pi'(s') = m]   \cdot \left(u(\theta, m(\theta), \omega) - u(\theta, m'(\theta),
 \omega)\right)\\
& -2 \cdot \sum_{\omega,s,s'} | \Pr[\omega_0 = \omega, S = s] \cdot \Pr[ S'=s'|S= s] -  \Pr[\omega_0 = \omega, S = s, S'=s']| \\
= &\sum_{\omega,s'} \Pr[\omega_0 = \omega, S'=s'] \cdot \Pr[\pi'(s') = m]   \cdot \left(u(\theta, m(\theta), \omega) - u(\theta, m'(\theta),
 \omega)\right)\\
& -2 \cdot \sum_{s} \Pr[S = s] \cdot  \sum_{s', \omega} \left| \Pr[S' = s', \omega_0 = \omega| S= s] - \Pr[S'=s'|S=s] \cdot \Pr[\omega_0 = \omega|S=s]\right| \\
\geq&  ~0- 2 \cdot \frac{\delta}{2}\\
 =& ~-\delta\\
\end{align*}

We also have for any $s', s ,m$ such that $Pr[S' = s',S = s] >0 $ and $\Pr[\pi'(s') = m] >0$, we have $\Pr[\pi(s) = m] > 0$. This implies $\EX_{s'}[\S'] \subseteq \EX^{\delta}_s[\S]$.
\end{proof}


\xhdr{Max-Support Policy.}
We can solve the following LP to check whether a particular menu $m_0 \in\A$ is signal-explorable given a particular realized signal $s_0\in\X$. In this LP, we represent a policy $\pi$ as a set of numbers
    $x_{m,s} = \Pr[\pi(s)=m]$,
for each menu $m\in \M$ and each feasible signal $s\in \X$.

\begin{figure}[H]
\begin{mdframed}
\begin{alignat*}{2}
 & \textbf{maximize }    x_{m_0,s_0}\  \\
&  \textbf{subject to: }\\
 & \sum_{\omega \in \varOmega, s \in \X} \Pr[\omega] \cdot \Pr[s | \omega] \cdot \left(u(\theta, m(\theta), \omega) - u(\theta, m'(\theta), \omega) + \delta\right) \cdot x_{m,s'} \geq 0  &\ & \forall m,m' \in \M, \theta \in \varTheta \\
                       & \sum_{m\in \M} x_{m,s} = 1,  \ &\ & \forall s \in \X \\
                       & x_{m,s} \geq 0,  \ &\ & \forall s \in \X, m\in \M \\
\end{alignat*}
\end{mdframed}
%\caption{LP}
\label{fig:nocc_lp}
\end{figure}

Since the constraints in this LP characterize any BIC recommendation policy, it follows that menu $m_0$ is $\delta$-signal-explorable given realized signal $s_0$ if and only if the LP has a positive solution. If such solution exists, define recommendation policy $\pi = \pi^{m_0,s_0}$ by setting $\Pr[\pi(s) = m] = x_{m,s}$ for all $m \in \M, s \in \X$. Then this is a $\delta$-BIC recommendation policy such that $\Pr[\pi(s_0) = m_0] > 0$.

\begin{definition}
Given a signal structure $\S$, a recommendation policy $\pi$ is called the $\delta$-max-support policy if $\forall s \in \X$  and $\delta$-signal-explorable menu $m\in \M$ given $s$, $\Pr[\pi(s) = m] > 0$.
\end{definition}

It is easy to see that we obtain $\delta$-max-support recommendation policy by averaging the $\pi^{m,s}$ policies define above.
Specifically, the following policy is a $\delta$-BIC and $\delta$-max-support policy.
\begin{align}
\label{eq:pimax2}
\pi^{max} = \frac{1}{|\X|} \sum_{s \in \X} \frac{1}{|\EX_s^{\delta}[\S]|} \sum_{m \in \EX_s^{\delta}[\S]} \pi^{m,s}.
\end{align}

\begin{comment}
Sometimes the term $\Pr[s'|\omega]$ in the above LP (i.e. the probability of seeing signal $s'$ given state to be $\omega$) is hard to compute. On the other hand, it could be easy to get a approximation of $\Pr[s'|\omega]$ as $p(s',\omega)$ such that $|p(s',w) -\Pr[s'|\omega]| \leq \beta$ for all $s',\omega$. Then we can solve a modified LP using $p(s',\omega)$ instead of $\Pr[s'|\omega]$:

\begin{figure}[H]
\begin{mdframed}
\begin{alignat*}{2}
 & \textbf{maximize }    x_{m,s}\  \\
&  \textbf{subject to: }\\
 & \sum_{\omega \in \varOmega, s' \in \X} \Pr[\omega] \cdot p(s,\omega) \cdot \left(u(\theta, m'(\theta), \omega) - u(\theta, m''(\theta), \omega)\right) \cdot x_{m',s'} \geq -\delta-\beta |\X|   &\ & \forall m',m'' \in \M, \theta \in \varTheta \\
& \sum_{m'\in \M} x_{m',s'} = 1,  \ &\ & \forall s' \in \X \\
& x_{m',s'} \geq 0,  \ &\ & \forall s' \in \X, m'\in \M \\
\end{alignat*}
\end{mdframed}
%\caption{LP}
\label{fig:nocc_lp_a}
\end{figure}

For the modified LP, we have the following claim:
\begin{claim}
\label{clm:nocc_lp_a}
For a given signal $s \in \X$, if a menu $m \in \M$ is $\delta$-signal-explorable, the above LP has a positive solution. When the LP has a positive solution, define $\pi^{m,s}$ as $\Pr[\pi^{m,s}(s') = m'] = x_{m',s', } \forall m' \in \M, s' \in \X$. Then $\pi^{m,s}$ is a single-round $(\delta+2\beta|\X|)$-BIC recommendation policy such that $\Pr[\pi^{m,s}(s) = m] > 0$
\end{claim}

\begin{definition}[Max-support policy]
Given a signal structure $\S$, a recommendation policy $\pi$ is called the $\delta$-max-support policy if $\forall s \in \X$  and $\delta$-signal-explorable menu $m\in \M$ given $s$, $\Pr[\pi(s) = m] > 0$.
\end{definition}

By Claim \ref{clm:nocc_lp_a}, we have the following claim.
\begin{claim}
\label{clm:pimax_nocc}
The following $\pi^{max}$ is a $(\delta+2\beta|\X|)$-BIC and $\delta$-max-support policy. Here $\M'$ is the set of menus with positive solutions in the LP mentioned in Claim \ref{clm:nocc_lp_a}.
\[
\pi^{max} = \frac{1}{|\X|} \sum_{s \in \X} \frac{1}{|\M'|} \sum_{m \in \M'} \pi^{m,s}.
\]
\end{claim}
\end{comment}


\xhdr{Maximal Exploration.}
Let us design a subroutine, called  MaxExplore, which outputs a sequence of $L$ menus. We are going to assume $L \geq \max_{m,s} \frac{B_m(\gamma_0)}{ \Pr[\pi^{max}(s)=m]}$. $\gamma_0$ is defined in Algorithm \ref{alg:nocc_main} of Section \ref{sec:private_main}.

The goal of this subroutine MaxExplore is to make sure that for any signal-explorable menu $m$, $m$ shows up at least $B_m(\gamma_0)$ times in the sequence with probability exactly 1. On the other hand, we want that the menu of each specific location in the sequence has marginal distribution same as $\pi^{max}$.

 \begin{algorithm}[H]
    \caption{Subroutine MaxExplore}
    	\label{alg:nocc_explore}
    \begin{algorithmic}[1]
	\STATE \textbf{Input:} signal $S$, signal structure $\S$.
	\STATE \textbf{Output:} a list of menus $\mu$
	\STATE Compute $\pi^{max}$ as per \eqref{eq:pimax2}.
	%\IF {$l \leq |\M| $}
		\STATE Initialize $Res = L$.
		\FOR {each menu $ m \in \M$}
			\STATE $C_m \leftarrow L \cdot \Pr[\pi^{max}(S) = m]$.
                     		\STATE Add $\lfloor C_m\rfloor$ copies of menu $m$ into list $\mu$.
			\STATE $Res \leftarrow Res -\lfloor C_m \rfloor $.
			\STATE $p^{Res}(m)\leftarrow  C_m -  \lfloor C_m\rfloor$
		\ENDFOR
		\STATE $p^{Res}(m) \leftarrow p^{Res}(m) / Res$, $\forall m \in \M$.
		\STATE Sample $Res$ many menus from distribution according to $p^{Res}$ independently and add these menus into $\mu$.
		\STATE Randomly permute the menus in $\mu$.
	\RETURN $\mu$.	
     \end{algorithmic}
\end{algorithm}

Similarly as the MaxExplore in Section \ref{sec:public}, we have the following claim.
\begin{claim}
\label{clm:maxexplore_nocc}
Given realized signal $S$, MaxExplore outputs a sequence of $L$ menus. Each menu in the sequence marginally distributed as $\pi^{max}$. For any menu $m$ such that $\Pr[\pi^{max} = m] >0$, $m$ shows up in the sequence at least $B_m(\gamma_0)$ times with probability exactly 1. MaxExplore runs in time polynomial in $L$, $|\M|$, $|\varOmega|$, $|\X|$ (size of the support of the signal).
\end{claim}

\xhdr{Menu Exploration.}
Given a menu $m$, a action-reward pair will be revealed to the algorithm after the round. Assuming the agent is following the menu, such action-reward pair is called a sample of the menu $m$. We use $D_m$ to the distribution of the samples. $D_m$ is a random variable depending on the state $\omega_0$. For a fixed state $\omega$, we use $D_m(\omega)$ to denote the distribution of the samples of menu $m$.

\begin{lemma}
\label{lem:deltam}
For any $\alpha > 0$, we can compute $\Delta_m$ which is a function of $B_m(\gamma) = O\left(\ln\left(\frac{1}{\gamma}\right)\right)$ samples of menu $m$ such that for any state $\omega$,
\[
\Pr[\Delta_m \neq D_m(\omega) | \omega_0 = \omega] \leq \gamma.
\]
\end{lemma}

\begin{proof}
Let $U$ be the union of the support of $D_m(\omega)$ for all $\omega \in \varOmega$. For each $u \in U$ ($u$ is just a sample of the menu), define $q(u,\omega) = \Pr_{v \sim D_m(\omega)}[v = u]$. Let $\delta_m$ be small enough such that for all $\omega, \omega'$ with $D_m(\omega) \neq D_m(\omega')$, there exists $u \in U$, such that $|q(u,\omega) - q(u,\omega')| > \delta_m$.

Now we compute $\Delta_m$ as following: Take $B_m(\gamma) = \frac{2}{\delta_m^2}\ln\left(\frac{2|U|}{\gamma}\right) $ samples and set $\hat{q}(u)$ as the empirical frequency of seeing $u$. And set $\Delta_m$ to be some $D_m(\omega)$ such that for all $u \in U$, $|q(u,\omega) - \hat{q}(u)| \leq \delta_m / 2$. Notice that if such $\omega$ exists, $\Delta_m$ will be unique. If no $\omega$ satisfies this, just pick $\Delta_m$ to be an arbitrary $D_m(\omega)$.

Now let's analyze $\Pr[\Delta_m \neq D_m(\omega)]$. Let's fixed the state $\omega_0 = \omega$. By Chernoff bound, for each $u \in U$,
\[
\Pr[|q(u,\omega) -\hat{q}(u)| > \delta_m/2] \leq 2\exp\left(-2 \cdot \left(\frac{\delta_m}{2}\right)^2 \cdot B_m(\gamma)\right) \leq \frac{\gamma}{|U|}.
\]
By union bound, with probability at least $1-\gamma$, we have for all$u \in U$, $|q(u,\omega) - \hat{q}(u)| \leq \delta_m / 2$. This implies $\Delta_m = D_m(\omega)$.
\end{proof}

\subsection{Main Recommendation Policy}
\label{sec:private_main}
In this subsection, we show our main recommendation policy which explores all the eventually-explorable menus and then recommends the agents the best menu given all history. We pick $L$ to be at least $\max_{m,s:\Pr[\pi(s)=m] >0} \frac{B_m(\gamma_0)}{ \Pr[\pi(s)=m]}$ for all $\pi$ that might be chosen as $\pi^{max}$ by Algorithm \ref{alg:nocc_main}.

 \begin{algorithm}[H]
    \caption{Main procedure for private types }
    	\label{alg:nocc_main}
    \begin{algorithmic}[1]
    	\STATE Initial signal $S_1 = \S_1= \perp$.
    	\STATE Set $\gamma_1 =\min\left(\frac{\delta^2}{16|\M|\log(|\varOmega|)},\left( \frac{\delta^2}{32|M|}\right)^2\right)$ and $\gamma_2 =  \frac{1}{T|\M|}$ and $\gamma_0=\min(\gamma_1,\gamma_2)$.
	\STATE Initial phase count $l = 1$.
	\FOR {$t=1$ to $T$}
		\IF {$l\leq |\M|$}
		\STATE \textbf{Exploration:}
		\IF {$t \equiv 1 \pmod L$}
			\STATE Start a new phase:
			
			\STATE Use the current $S_l$ and $\S_l$ to compute a list of $L$ menus $\mu \leftarrow $ MaxExplore($S_l, \S_l$).
		\ENDIF
		\STATE Suggest menu $\mu [ (t-1) \mod L + 1]$ to the agent.
		\IF {$t \equiv 0 \pmod L$}
			\STATE End of a phase:
			\STATE For each explored menu $m$ in the previous phase, use $B_m(\gamma_1)$ samples to compute $\Delta_m$ stated in Lemma \ref{lem:deltam}.
			\STATE If there does not exist a state $w$ which is consistent with $\Delta_m$ ($\Delta_m = D_m(\omega)$) for all explored menu $m$, pick an arbitrary state $\omega$ and set $\Delta_m \leftarrow D_m(\omega)$ for all explored menu $m$. This step just make sure the number of signals is bounded by $|\varOmega|$.
			\STATE $l \leftarrow l + 1$.
			\STATE Set $S_l$ to be the collection of $\Delta_m$'s for all explored menu $m$.
		\ENDIF
	\ELSE
		\STATE \textbf{Exploitation:}
		\STATE If this is the first exploitation round, for each explored menu $m$ in the exploration, use $B_m(\gamma_2)$ samples to compute $\Delta_m$ stated in Lemma \ref{lem:deltam}. Set $S_l$ to be the collection of $\Delta_m$'s for all explored menu $m$.
		\STATE Suggest the menu which consists of the best action of each type conditioned on $S_l$ and the prior.
	\ENDIF
	\ENDFOR
     \end{algorithmic}
\end{algorithm}

First of all, it's easy to check by Claim \ref{clm:maxexplore_nocc} that for each agent, it is $\delta$-BIC to follow the recommended action if previous agents all follow the recommended actions. Therefore we have the following claim.
\begin{claim}
\label{clm:nocc_BIC}
Algorithm \ref{alg:nocc_main} is $\delta$-BIC.
\end{claim}


\begin{lemma}
\label{lem:exp_nocc}
For any $l > 0$, assume Algorithm \ref{alg:nocc_main} has at least $\min(l, |\M|)$ phases.
For a given state $\omega$, if a menu $m$ can be explored by a BIC recommendation policy $\pi$ at round $l$ (i.e. $ \Pr[\pi^l= m]> 0$), then such menu is guaranteed to be explored $B_m$ times by Algorithm \ref{alg:nocc_main} by the end of phase $\min(l, |\M|)$.
\end{lemma}

\begin{proof}
The proof is similar to Lemma \ref{lem:exp_public}. We prove by induction on $l$ for $l \leq |\M|$.

%Base case $l=1$ is trivial by Claim \ref{clm:maxexplore}. Assuming the lemma is correct for $l-1$, let's prove it's correct for $l$.

Let $S$ be the signal of Algorithm \ref{alg:nocc_main} in phase $l$. Let $S'$ be the history of $\pi$ in the first $l-1$ rounds. More precisely, $S' = R, H_1,...,H_{l-1}$. Here $R$ is the internal randomness of $\pi$ and $H_t = (M_t, A_t,u(\Theta_t, M_t(\Theta_t), \omega_0))$ is the menu and the action-reward pair in round $t$ of $\pi$.

Let $\M'$ to be the set of menus explored in the first $l-1$ phases of Algorithm \ref{alg:nocc_main}. By the induction hypothesis, we have $\forall t\in[l-1]$, $M_t \subseteq \M'$.

First of all, we have
\[
I(S'; \omega_0| S) = I(R,H_1,...,H_{l-1}; \omega_0| S)  = I(R; \omega_0| S) + I(H_1,...,H_{l-1}; \omega_0|S, R) = I(H_1,...,H_{l-1}; \omega_0|S, R).
\]

By the chain rule of mutual information, we have
\[
 I(H_1,...,H_{l-1}; \omega_0|S, R) = I(H_1;\omega_0|S,R) + I(H_2;\omega_0|S, R ,H_1) + \cdots + I(H_{l-1}; \omega_0|S,R,H_1,...,H_{l-2}).
\]

For all $t \in [l-1]$, we have
\begin{align*}
&I(H_t; \omega_0|S,R,H_1,...,H_{t-1}) \\
=& I(M_t,A_t, u(\Theta_t, M_t(\Theta_t), \omega_0); \omega_0|S,R,H_1,...,H_{t-1}) \\
=& I(A_t, u(\Theta_t, M_t(\Theta_t), \omega_0); \omega_0 | S,R,H_1,...,H_{t-1}, M_t)\\
\leq& I(D_{M_t}; \omega_0|S,R,H_1,...,H_{t-1},M_t). \\
\end{align*}
The second last step comes from the fact that $M_t$ is a deterministic function of $R,H_1,...,H_{t-1}$. The last step comes from the fact that $(A_t,u(\Theta_t, M_t(\Theta_t), \omega_0))$ is independent with $\omega_0$ given $D_{M_t}$.

Then we have
\begin{align*}
& I(D_{M_t}; \omega_0|S,R,H_1,...,H_{t-1},M_t)\\
=& \sum_{m \in \M'} \Pr[M_t = m] \cdot I(D_m;\omega_0 | S,R,H_1,...,H_{t-1},M_t = m)\\
\leq& \sum_{m \in M'} \Pr[M_t = m] \cdot I(D_m;\omega_0| \Delta_m, M_t =m).\\
\leq& \sum_{m \in M'} \Pr[M_t = m] \cdot H(D_m| \Delta_m, M_t =m).
\end{align*}
The last step comes from the fact that $I(D_m; (S\backslash \Delta_m),R,H_1,...,H_{t-1}|\omega_0, \Delta_m, M_t =m) = 0$. By Lemma \ref{lem:deltam}, we know that $\Pr[D_m \neq \Delta_m|M_t = m] \leq \gamma_1$. By Fano's inequality, we have
\[
H(D_m| \Delta_m, M_t =m) \leq H(\gamma_1) + \gamma_1 \log(|\varOmega| - 1) \leq 2\sqrt{\gamma_1} + \gamma_1 \log(|\varOmega| - 1)\leq \frac{\delta^2}{16|\M|}+\frac{\delta^2}{16|\M|}  = \frac{\delta^2}{8|\M|}.
\]

Therefore we have
\[
I(H_t; \omega_0|S,R,H_1,...,H_{t-1}) \leq \frac{\delta^2}{8|\M|}, \forall t \in [l-1].
\]
Then we get
\[
I(S'; \omega_0 | S) \leq \delta^2/8.
\]

By Lemma \ref{lem:ainfomono}, we know that $\EX_{s'}[\S'] \subseteq \EX^{\delta}_s[\S]$. By Claim \ref{clm:maxexplore_nocc}, we know that phase $l$ will explore menu $m$ at least $B_m(\gamma_0)$ times.

When $l > |\M|$, the proof follows from the same argument as the last paragraph of the proof of Lemma \ref{lem:exp_public}.
\end{proof}

\begin{corollary}[Restatement of Theorem \ref{thm:private_nocc}]
\label{cor:private_nocc}
For any $\delta > 0$, we have a $\delta$-BIC recommendation policy of $T$ rounds with expected total reward at least $\left(T - C\cdot \log(T) \right) \cdot \OPT$ for some constant $C$ which does not depend on $T$.
\end{corollary}

\begin{proof}

First of all, by Claim \ref{clm:nocc_BIC}, Algorithm \ref{alg:nocc_main} is $\delta$-BIC.

By Lemma \ref{lem:exp_nocc}, for each state $\omega$, Algorithm \ref{alg:nocc_main} explores all the eventually-explorable menus (i.e. $\MExp_{\omega}$) for by the end of $|\M|$ phases.

After that, by Lemma \ref{lem:deltam} and $\gamma_2 = \frac{1}{T|\M|}$, for a fixed state $\omega$, we know that with probability $1- 1/T$, $\delta_m = D_m$ for all $m \in \MExp_{\omega}$. In this case, the agent of type $\theta$ gets expected reward at least $u(\theta,m^*(\theta),\omega)$ where menu $m^* =\arg\max_{m \in \MExp_{\omega}} \sum_{\theta \in \varTheta} \Pr[\theta] \cdot u(\theta, m(\theta), \omega)$. Taking average over types, the expected reward per round should be at least $(1-1/T) \cdot \max_{m \in \MExp_{\omega}} \sum_{\theta \in \varTheta} \Pr[\theta] \cdot u(\theta, m(\theta), \omega)$.

We know that the expected number of rounds of the first $|\M|$ phases is $|\M| \cdot L = O(\ln(T))$. Therefore, Algorithm \ref{alg:nocc_main} has expected total reward at least $T \cdot \OPT- T \cdot (1/T) - O(\ln(T)) = T\cdot \OPT - O(\ln(T))$.

\end{proof}



%!TEX root = main.tex

\section{Comparative Statics for Explorability}
\label{sec:statics}

\newcommand{\pairs}{\mP_\omega}
\newcommand{\pairsPub}{\pairs^{\term{pub}}}
\newcommand{\pairsPri}{\pairs^{\term{pri}}}
\newcommand{\support}{\term{support}}

We discuss how the set of all explorable type-action pairs (\emph{explorable set}) is affected by the model choice and the diversity of types. In what follows, let $\pairs$ be the explorable set for state $\omega\in\varOmega$.

\xhdr{Explorability and the model choice.}
Fix an instance of Bayesian Persuasion. Recall from Section~\ref{sec:reported} that the explorable set is not affected if we switch from public types to reported types. 

What if we switch from public types to private types? Let $\pairsPub$ and $\pairsPri$ denote $\pairs$ for public types and private types, respectively.\footnote{For private types, rewards for all type-action pairs in $\pairsPri$ is an ``upper bound" on the information available to the principal. The principal may know ``less" than that: it does not learn the agent type when the recommended menu maps multiple types to the chosen action.}
Then we have: 
 
\begin{claim}
$\pairsPri \subseteq \pairsPub$.
\end{claim}

The idea in the proof is that one can simulate any BIC recommendation policy for private types with a BIC recommendation policy for public types; we omit the details from this version.%

Let us prove an example when $\pairsPri$ is a strict subset of $\pairsPub$.

\begin{example}
\label{exp:simple}
There are two states, two types and two actions: 
    $\varOmega = \varTheta = \A = \{0,1\}$. 
States and types are drawn uniformly at random:
    $\Pr[\omega =0] =\Pr[\theta =0] = \tfrac12$. 
Utilities are defined in the following table:\\
\begin{table}[H]
\centering
\begin{tabular}{|c||c|c|}
\hline
&$a=0$&$a=1$\\
\hline
\hline
$\theta = 0$& $u = 3$ & $u =4$\\
\hline
$\theta = 1$& $u = 2$ & $u =0$\\
\hline
\end{tabular}
\quad
\begin{tabular}{|c||c|c|}
\hline
&$a=0$&$a=1$\\
\hline
\hline
$\theta = 0$& $u = 2$ & $u =0$\\
\hline
$\theta = 1$& $u = 3$ & $u =4$\\
\hline
\end{tabular}
\caption{Utilities $u(\theta,a,\omega)$ when $\omega =0 $ and $\omega = 1$.}
\end{table}
\end{example}

It is easy to check that action 0 is preferred by both types when agents have no information about the state. When types are public or reported, the principal learns the state after the first round and then the action of utility 4 can be explored in later rounds. For example, $\A_{0,0} = \{0,1\}$.

For private types, in round $1$ the principal can only recommend a menu that recommends action 0 for both types. Samples from this menu do not convey any information about the state, as they are distributed the same in both states. Therefore, action $1$ is not eventually-explorable, for either type and either state.  Thus:

\begin{claim}
In Example \ref{exp:simple}, $\pairsPri$ is a strict subset of $\pairsPub$.
\end{claim}

\xhdr{Explorability and diversity of agent types.}
Let us discuss whether and how the type distribution affects the explorable set. Fix an instance of Bayesian Exploration with type distribution $\DT$. Let us see how the explorable set changes if we change $\DT$ to some other distribution $\DT'$. 

Let $\pairs$ and $\pairs'$ be the corresponding explorable sets, for each state $\omega$. Let $\support(\DT)$ be the support set of distribution $\DT$, \ie the set of all feasible agent types according to $\DT$.

For public types, we show that the explorable set is determined by the support set of $\DT$, and can only increase if the support set increases:

\begin{claim}\label{cl:statics-diversity-public}
Consider Bayesian Exploration with public types. Then:
\begin{OneLiners}
\item[(a)] if $\support(\DT)=\support(\DT')$ then $\pairs=\pairs'$.
\item[(b)] if $\support(\DT)\subset \support(\DT')$ then $\pairs\subseteq \pairs'$.
\end{OneLiners}
\end{claim}

\begin{proof}[Proof Sketch]
We follow the steps of the proof of Lemma \ref{lem:exp_public}. The idea is that the proof carries through even if $\pi$ is a BIC recommendation policy for the less diverse problem instance, \ie an instance with type distribution $\DT'$ such that $\support(\DT')\subset \support(\DT)$. Then for a given state $\omega$, Algorithm \ref{alg:public_main} for the original distribution $\DT$ instance explores all type-action pairs in $\pairs'$.
\end{proof}

The conclusions in Claim~\ref{cl:statics-diversity-public} apply to reported types, too. This is because explorable sets are the same for public and reported types.

For private types, the situation is more complicated. More types can help for some problem instances. For example, if different types have disjoint sets of available actions (more formally: say, disjoint sets of actions with positive rewards) then we are essentially back to the case of reported types, and the conclusions in Claim~\ref{cl:statics-diversity-public} apply. On the other hand, 
we can use Example \ref{exp:simple} to show that more types can hurt explorability when types are private. Recall that in this example, for private types only action 0 can be recommended. Now consider a less diverse instance in which only type 0 appears. After one agent in that type chooses action 0, the state is reviewed to the principal. For example, when the state $\omega = 0$, action $1$ can be recommended to future agents. This shows that,  in this example, explorable set increases when we have fewer types. 

\bibliographystyle{alpha}
\bibliography{references,bib-abbrv,bib-slivkins,bib-bandits,bib-AGT,bib-ML}

\appendix
%!TEX root = main.tex
\section{Basics of Information Theory}
\label{app:info-theory}

We briefly review some standard facts and definitions from information theory which are used in proofs. For a more detailed introduction, see \cite{CK11}. Throughout, $X,Y,Z,W$ are random variables that take values in an arbitrary domain (not necessarily $\R$).

\xhdr{Entropy.}
The fundamental notion is \emph{entropy} of a random variable. In particular, if $X$ has finite support, its entropy is defined as
\[ H(X) = \textstyle - \sum_{x} p(x)\cdot  \log p(x),
\quad\text{where } p(x) = \Pr[X = x]. \]
(Throughout this paper, we use $\log$ to refer to the base $2$ logarithm and use $\ln$ to refer to the natural logarithm.) If $X$ is drawn from Bernoulli distribution with $\E[X]=p$, then
    \[ H(p) = -(p\log p + (1-p)(\log(1-p)). \]

The conditional entropy of $X$ given event $E$ is the entropy of the conditional distribution $(X|E)$:
\[ H(X|E) = \textstyle - \sum_{x} p(x)\cdot  \log p(x),
\quad\text{where } p(x) = \Pr[X = x | E]. \]

The \emph{conditional entropy} of $X$ given $Y$ is
\[ H(X|Y)
    := \E_y[H(X|Y = y)]
    = \textstyle \sum_{y} \Pr[Y=y]\cdot H(X|Y = y). \]
Note that $H(X|Y) = H(X)$ if $X$ and $Y$ are independent.

We are sometimes interested in the entropy of a tuple of random variables, such as $(X,Y,Z)$. To simplify notation, we will write $H(X,Y,Z)$ instead $H((X,Y,Z))$, and similarly in other information-theoretic notation. With this ado, we can formulate the \emph{Chain Rule} for entropy:
\begin{align}\label{app:info-entropy-chain-rule}
 H(X,Y) = H(X) + H(Y|X). 
\end{align}


We also use the following fundamental fact about entropy:

\begin{lemma}[Fano's Inequality]
Let $X,Y,\hat{X}$ be random variables such that $\hat{X}$ is a deterministic function of $Y$. (Informally, $\hat{X}$ is an approximate version of $X$ derived from signal $Y$.) Let $E = \{ \hat{X} \neq X \}$ be the ``error event". Then
    \[ H(X|Y) \leq H(E) + \Pr[E] \cdot (\log(|\X|-1), \]
where $\X$ denotes the support set of $X$.
\end{lemma} 

\xhdr{Mutual information.}
The \emph{mutual information} between $X$ and $Y$ is
\[ I(X;Y) := H(X) - H(X|Y) = H(Y) - H(Y|X).\]
The \emph{conditional mutual information} between $X$ and $Y$ given $Z$ is
\[ I(X;Y|Z) := H(X|Z) - H(X|Y,Z) = H(Y|Z) - H(Y|X,Z).\]
Note that $I(X;Y|Z) = I(X;Y)$ if $X$ and $Z$ are conditionally independent given $Y$, and $Y$ and $Z$ are conditionally independent given $X$.

Some of the fundamental properties of conditional mutual information are as follows:
\begin{align}
I(X,Y;Z|W) &= I(X;Z|W) + I(Y;Z|W,X) \\
I(X;Y|Z) &\geq I(X;Y|Z,W) \qquad\text{if $I(Y;W|X,Z) = 0$} \\
I(X;Y|Z) &\leq I(X;Y|Z,W) \qquad\text{if $I(Y;W|Z) = 0$} 
\end{align}

\xhdr{KL-divergence.}
The \emph{Kullback-Leibler divergence} (a.k.a., \emph{KL-divergence}) between random variables $X$ and $Y$ is defined as
\[ \DKL(X\| Y) = \sum_x \Pr[X = x] 
    \cdot \log\left( \frac{\Pr[X = x]}{\Pr[Y = x]} \right) .\]
Note that the definition is not symmetric, in the sense that in general 
    $\DKL(X\| Y)\neq \DKL(Y\| X)$.

KL-divergence can be related to conditional mutual information as follows:
\begin{align}
I(X;Y|Z) 
    &= \mathbb{E}_{x,z}\left[ \; \DKL((Y|X = x, Z=z)\|(Y|Z=z)) \; \right] \nonumber \\
    &= \sum_{x,z} \Pr[X=x,Z=z]\;\ \DKL((Y|X = x, Z=z)\|(Y|Z=z)). 
\end{align}
Here $(Y|E)$ denotes the conditional distribution of $Y$ given event $E$.

We also use \emph{Pinsker Inequality}:
\begin{align}
\sum_x | \Pr[X=x] - \Pr[Y=x]| \leq \sqrt{2 \ln(2)\, \DKL(X\|Y)}.
\end{align}


\end{document}
