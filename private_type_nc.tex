%!TEX root = main.tex

\section{Bayesian Exploration with Private Types}
\label{sec:private_nc}

Our recommendation policy for private types satisfies a relaxed version of the BIC property, called \emph{$\delta$-BIC}, where the right-hand side in \eqref{eq:model-BIC-menus} is $-\delta$ for some fixed $\delta>0$. We assume a more permissive behavioral model in which agents follow recommendations of such policy. Throughout this section, $\OPT = \OPTpri$.

 
\begin{theorem}
\label{thm:private_nocc}
For any $\delta > 0$, we have a $\delta$-BIC recommendation policy of $T$ rounds with expected total reward at least $\left(T - C\cdot \log(T) \right) \cdot \OPT$ for some constant $C$ which does not depend on $T$.
\end{theorem}

\ascomment{stopped here.}

\subsection{Single-round Exploration}
\label{sec:private_single}

In this subsection, we consider a single round of the Bayesian exploration.

\begin{definition}
Consider a single-round of Bayesian exploration when the principal has signal $S$ from signal structure $\S$. For any $\delta \geq 0$, a menu $m \in \M$ is called $\delta$-signal-explorable, for a given signal $s$, if there exists a single-round $\delta$-BIC recommendation policy $\pi$ such that $\Pr[\pi(s) = m] > 0$. The set of all such menus is denoted as $\EX^{\delta}_s[\S]$. The $\delta$-signal-explorable set is defined as $\EX^{\delta}[\S] = \EX^{\delta}_S[\S]$. We omit $\delta$ in $\EX^{\delta}[\S]$ when $\delta = 0$.
\end{definition}

\xhdr{Approximate Information Monotonicity.}
In the following definition, we define a way to compare two signals approximately.
\begin{definition}
Let $S$ and $S'$ be two random variables. We say random variable $S$ is $\alpha$-approximately informative as random variable $S'$ about state $\omega_0$ if $I(S' ; \omega_0|S) = \alpha$.
\end{definition}

\begin{lemma}
\label{lem:ainfomono}
Let $S$ and $S'$ be two random variables and $\S$ and $\S'$ be their signal structures. If $S$ is $(\delta^2/8)$-approximately informative as $S'$ about state $\omega_0$ (i.e. $I(S' ; \omega_0|S) \leq \delta^2/8$), then $\EX_{s'}[\S'] \subseteq \EX^{\delta}_s[\S]$  for all $s' ,s$ such that $\Pr[S= s, S'= s'] > 0$.
\end{lemma}

\begin{proof}
We have
\[
\sum_{s} \Pr[S=s] \cdot \DKL\left(S'\omega_0|S=s \| (S'|S=s) \times (\omega_0|S=s) \right) = I(S' ; \omega_0|S) \leq \delta^2/8.
\]
By Pinsker's inequality, we have
\begin{align*}
       &\sum_{s} \Pr[S = s] \cdot  \sum_{s', \omega} \left| \Pr[S' = s', \omega_0 = \omega| S= s] - \Pr[S'=s'|S=s] \cdot \Pr[\omega_0 = \omega|S=s]\right| \\
\leq & \sum_{s} \Pr[S=s] \cdot \sqrt{2 \DKL\left(S'\omega_0|S=s \| (S'|S=s) \times (\omega_0|S=s) \right)  } \\
\leq &  \sqrt{2 \sum_{s} \Pr[S=s] \cdot  \DKL\left(S'\omega_0|S=s \| (S'|S=s) \times (\omega_0|S=s) \right) } \\
\leq &\delta /2. \\
\end{align*}

Consider any BIC recommendation policy $\pi'$ for signal structure $\S'$. We construct $\pi$ for signature structure $\S$ by setting $\Pr[\pi(s) = m] = \sum_{s'} \Pr[\pi'(s') = m] \cdot Pr[S' = s'|S = s]$.

Now we check $\pi$ is $\delta$-BIC. For any $m,m' \in \M$ and $\theta \in \varTheta$,
\begin{align*}
& \sum_{\omega,s} \Pr[\omega_0= \omega] \cdot \Pr[S = s | \omega_0 = \omega] \cdot \left(u(\theta, m(\theta), \omega) - u(\theta, m'(\theta), \omega)\right) \cdot  \Pr[\pi(s) = m] \\
=&\sum_{\omega,s,s'} \Pr[\omega_0 = \omega, S = s] \cdot \Pr[ S'=s'|S= s] \cdot \Pr[\pi'(s') = m]   \cdot \left(u(\theta, m(\theta), \omega) - u(\theta, m'(\theta), \omega)\right)\\
\geq &\sum_{\omega,s,s'} \Pr[\omega_0 = \omega, S = s, S'=s'] \cdot \Pr[\pi'(s') = m]   \cdot \left(u(\theta, m(\theta), \omega) - u(\theta, m'(\theta),
 \omega)\right)\\
& -2 \cdot \sum_{\omega,s,s'} | \Pr[\omega_0 = \omega, S = s] \cdot \Pr[ S'=s'|S= s] -  \Pr[\omega_0 = \omega, S = s, S'=s']| \\
= &\sum_{\omega,s'} \Pr[\omega_0 = \omega, S'=s'] \cdot \Pr[\pi'(s') = m]   \cdot \left(u(\theta, m(\theta), \omega) - u(\theta, m'(\theta),
 \omega)\right)\\
& -2 \cdot \sum_{s} \Pr[S = s] \cdot  \sum_{s', \omega} \left| \Pr[S' = s', \omega_0 = \omega| S= s] - \Pr[S'=s'|S=s] \cdot \Pr[\omega_0 = \omega|S=s]\right| \\
\geq&  ~0- 2 \cdot \frac{\delta}{2}\\
 =& ~-\delta\\
\end{align*}

We also have for any $s', s ,m$ such that $Pr[S' = s',S = s] >0 $ and $\Pr[\pi'(s') = m] >0$, we have $\Pr[\pi(s) = m] > 0$. This implies $\EX_{s'}[\S'] \subseteq \EX^{\delta}_s[\S]$.
\end{proof}


\xhdr{Max-Support Policy.}
We can solve the following LP to check whether a particular menu $m_0 \in\A$ is signal-explorable given a particular realized signal $s_0\in\X$. In this LP, we represent a policy $\pi$ as a set of numbers
    $x_{m,s} = \Pr[\pi(s)=m]$,
for each menu $m\in \M$ and each feasible signal $s\in \X$.

\begin{figure}[H]
\begin{mdframed}
\begin{alignat*}{2}
 & \textbf{maximize }    x_{m_0,s_0}\  \\
&  \textbf{subject to: }\\
 & \sum_{\omega \in \varOmega, s \in \X} \Pr[\omega] \cdot \Pr[s | \omega] \cdot \left(u(\theta, m(\theta), \omega) - u(\theta, m'(\theta), \omega) + \delta\right) \cdot x_{m,s'} \geq 0  &\ & \forall m,m' \in \M, \theta \in \varTheta \\
                       & \sum_{m\in \M} x_{m,s} = 1,  \ &\ & \forall s \in \X \\
                       & x_{m,s} \geq 0,  \ &\ & \forall s \in \X, m\in \M \\
\end{alignat*}
\end{mdframed}
%\caption{LP}
\label{fig:nocc_lp}
\end{figure}

Since the constraints in this LP characterize any BIC recommendation policy, it follows that menu $m_0$ is $\delta$-signal-explorable given realized signal $s_0$ if and only if the LP has a positive solution. If such solution exists, define recommendation policy $\pi = \pi^{m_0,s_0}$ by setting $\Pr[\pi(s) = m] = x_{m,s}$ for all $m \in \M, s \in \X$. Then this is a $\delta$-BIC recommendation policy such that $\Pr[\pi(s_0) = m_0] > 0$.

\begin{definition}
Given a signal structure $\S$, a recommendation policy $\pi$ is called the $\delta$-max-support policy if $\forall s \in \X$  and $\delta$-signal-explorable menu $m\in \M$ given $s$, $\Pr[\pi(s) = m] > 0$.
\end{definition}

It is easy to see that we obtain $\delta$-max-support recommendation policy by averaging the $\pi^{m,s}$ policies define above.
Specifically, the following policy is a $\delta$-BIC and $\delta$-max-support policy.
\begin{align}
\label{eq:pimax2}
\pi^{max} = \frac{1}{|\X|} \sum_{s \in \X} \frac{1}{|\EX_s^{\delta}[\S]|} \sum_{m \in \EX_s^{\delta}[\S]} \pi^{m,s}.
\end{align}

\begin{comment}
Sometimes the term $\Pr[s'|\omega]$ in the above LP (i.e. the probability of seeing signal $s'$ given state to be $\omega$) is hard to compute. On the other hand, it could be easy to get a approximation of $\Pr[s'|\omega]$ as $p(s',\omega)$ such that $|p(s',w) -\Pr[s'|\omega]| \leq \beta$ for all $s',\omega$. Then we can solve a modified LP using $p(s',\omega)$ instead of $\Pr[s'|\omega]$:

\begin{figure}[H]
\begin{mdframed}
\begin{alignat*}{2}
 & \textbf{maximize }    x_{m,s}\  \\
&  \textbf{subject to: }\\
 & \sum_{\omega \in \varOmega, s' \in \X} \Pr[\omega] \cdot p(s,\omega) \cdot \left(u(\theta, m'(\theta), \omega) - u(\theta, m''(\theta), \omega)\right) \cdot x_{m',s'} \geq -\delta-\beta |\X|   &\ & \forall m',m'' \in \M, \theta \in \varTheta \\
& \sum_{m'\in \M} x_{m',s'} = 1,  \ &\ & \forall s' \in \X \\
& x_{m',s'} \geq 0,  \ &\ & \forall s' \in \X, m'\in \M \\
\end{alignat*}
\end{mdframed}
%\caption{LP}
\label{fig:nocc_lp_a}
\end{figure}

For the modified LP, we have the following claim:
\begin{claim}
\label{clm:nocc_lp_a}
For a given signal $s \in \X$, if a menu $m \in \M$ is $\delta$-signal-explorable, the above LP has a positive solution. When the LP has a positive solution, define $\pi^{m,s}$ as $\Pr[\pi^{m,s}(s') = m'] = x_{m',s', } \forall m' \in \M, s' \in \X$. Then $\pi^{m,s}$ is a single-round $(\delta+2\beta|\X|)$-BIC recommendation policy such that $\Pr[\pi^{m,s}(s) = m] > 0$
\end{claim}

\begin{definition}[Max-support policy]
Given a signal structure $\S$, a recommendation policy $\pi$ is called the $\delta$-max-support policy if $\forall s \in \X$  and $\delta$-signal-explorable menu $m\in \M$ given $s$, $\Pr[\pi(s) = m] > 0$.
\end{definition}

By Claim \ref{clm:nocc_lp_a}, we have the following claim.
\begin{claim}
\label{clm:pimax_nocc}
The following $\pi^{max}$ is a $(\delta+2\beta|\X|)$-BIC and $\delta$-max-support policy. Here $\M'$ is the set of menus with positive solutions in the LP mentioned in Claim \ref{clm:nocc_lp_a}.
\[
\pi^{max} = \frac{1}{|\X|} \sum_{s \in \X} \frac{1}{|\M'|} \sum_{m \in \M'} \pi^{m,s}.
\]
\end{claim}
\end{comment}


\xhdr{Maximal Exploration.}
Let us design a subroutine, called  MaxExplore, which outputs a sequence of $L$ menus. We are going to assume $L \geq \max_{m,s} \frac{B_m(\gamma_0)}{ \Pr[\pi^{max}(s)=m]}$. $\gamma_0$ is defined in Algorithm \ref{alg:nocc_main} of Section \ref{sec:private_main}.

The goal of this subroutine MaxExplore is to make sure that for any signal-explorable menu $m$, $m$ shows up at least $B_m(\gamma_0)$ times in the sequence with probability exactly 1. On the other hand, we want that the menu of each specific location in the sequence has marginal distribution same as $\pi^{max}$.

 \begin{algorithm}[H]
    \caption{Subroutine MaxExplore}
    	\label{alg:nocc_explore}
    \begin{algorithmic}[1]
	\STATE \textbf{Input:} signal $S$, signal structure $\S$.
	\STATE \textbf{Output:} a list of menus $\mu$
	\STATE Compute $\pi^{max}$ as per \eqref{eq:pimax2}.
	%\IF {$l \leq |\M| $}
		\STATE Initialize $Res = L$.
		\FOR {each menu $ m \in \M$}
			\STATE $C_m \leftarrow L \cdot \Pr[\pi^{max}(S) = m]$.
                     		\STATE Add $\lfloor C_m\rfloor$ copies of menu $m$ into list $\mu$.
			\STATE $Res \leftarrow Res -\lfloor C_m \rfloor $.
			\STATE $p^{Res}(m)\leftarrow  C_m -  \lfloor C_m\rfloor$
		\ENDFOR
		\STATE $p^{Res}(m) \leftarrow p^{Res}(m) / Res$, $\forall m \in \M$.
		\STATE Sample $Res$ many menus from distribution according to $p^{Res}$ independently and add these menus into $\mu$.
		\STATE Randomly permute the menus in $\mu$.
	\RETURN $\mu$.	
     \end{algorithmic}
\end{algorithm}

Similarly as the MaxExplore in Section \ref{sec:public}, we have the following claim.
\begin{claim}
\label{clm:maxexplore_nocc}
Given realized signal $S$, MaxExplore outputs a sequence of $L$ menus. Each menu in the sequence marginally distributed as $\pi^{max}$. For any menu $m$ such that $\Pr[\pi^{max} = m] >0$, $m$ shows up in the sequence at least $B_m(\gamma_0)$ times with probability exactly 1. MaxExplore runs in time polynomial in $L$, $|\M|$, $|\varOmega|$, $|\X|$ (size of the support of the signal).
\end{claim}

\xhdr{Menu Exploration.}
Given a menu $m$, a action-reward pair will be revealed to the algorithm after the round. Assuming the agent is following the menu, such action-reward pair is called a sample of the menu $m$. We use $D_m$ to the distribution of the samples. $D_m$ is a random variable depending on the state $\omega_0$. For a fixed state $\omega$, we use $D_m(\omega)$ to denote the distribution of the samples of menu $m$.

\begin{lemma}
\label{lem:deltam}
For any $\alpha > 0$, we can compute $\Delta_m$ which is a function of $B_m(\gamma) = O\left(\ln\left(\frac{1}{\gamma}\right)\right)$ samples of menu $m$ such that for any state $\omega$,
\[
\Pr[\Delta_m \neq D_m(\omega) | \omega_0 = \omega] \leq \gamma.
\]
\end{lemma}

\begin{proof}
Let $U$ be the union of the support of $D_m(\omega)$ for all $\omega \in \varOmega$. For each $u \in U$ ($u$ is just a sample of the menu), define $q(u,\omega) = \Pr_{v \sim D_m(\omega)}[v = u]$. Let $\delta_m$ be small enough such that for all $\omega, \omega'$ with $D_m(\omega) \neq D_m(\omega')$, there exists $u \in U$, such that $|q(u,\omega) - q(u,\omega')| > \delta_m$.

Now we compute $\Delta_m$ as following: Take $B_m(\gamma) = \frac{2}{\delta_m^2}\ln\left(\frac{2|U|}{\gamma}\right) $ samples and set $\hat{q}(u)$ as the empirical frequency of seeing $u$. And set $\Delta_m$ to be some $D_m(\omega)$ such that for all $u \in U$, $|q(u,\omega) - \hat{q}(u)| \leq \delta_m / 2$. Notice that if such $\omega$ exists, $\Delta_m$ will be unique. If no $\omega$ satisfies this, just pick $\Delta_m$ to be an arbitrary $D_m(\omega)$.

Now let's analyze $\Pr[\Delta_m \neq D_m(\omega)]$. Let's fixed the state $\omega_0 = \omega$. By Chernoff bound, for each $u \in U$,
\[
\Pr[|q(u,\omega) -\hat{q}(u)| > \delta_m/2] \leq 2\exp\left(-2 \cdot \left(\frac{\delta_m}{2}\right)^2 \cdot B_m(\gamma)\right) \leq \frac{\gamma}{|U|}.
\]
By union bound, with probability at least $1-\gamma$, we have for all$u \in U$, $|q(u,\omega) - \hat{q}(u)| \leq \delta_m / 2$. This implies $\Delta_m = D_m(\omega)$.
\end{proof}

\subsection{Main Recommendation Policy}
\label{sec:private_main}
In this subsection, we show our main recommendation policy which explores all the eventually-explorable menus and then recommends the agents the best menu given all history. We pick $L$ to be at least $\max_{m,s:\Pr[\pi(s)=m] >0} \frac{B_m(\gamma_0)}{ \Pr[\pi(s)=m]}$ for all $\pi$ that might be chosen as $\pi^{max}$ by Algorithm \ref{alg:nocc_main}.

 \begin{algorithm}[H]
    \caption{Main procedure for private types }
    	\label{alg:nocc_main}
    \begin{algorithmic}[1]
    	\STATE Initial signal $S_1 = \S_1= \perp$.
    	\STATE Set $\gamma_1 =\min\left(\frac{\delta^2}{16|\M|\log(|\varOmega|)},\left( \frac{\delta^2}{32|M|}\right)^2\right)$ and $\gamma_2 =  \frac{1}{T|\M|}$ and $\gamma_0=\min(\gamma_1,\gamma_2)$.
	\STATE Initial phase count $l = 1$.
	\FOR {$t=1$ to $T$}
		\IF {$l\leq |\M|$}
		\STATE \textbf{Exploration:}
		\IF {$t \equiv 1 \pmod L$}
			\STATE Start a new phase:
			
			\STATE Use the current $S_l$ and $\S_l$ to compute a list of $L$ menus $\mu \leftarrow $ MaxExplore($S_l, \S_l$).
		\ENDIF
		\STATE Suggest menu $\mu [ (t-1) \mod L + 1]$ to the agent.
		\IF {$t \equiv 0 \pmod L$}
			\STATE End of a phase:
			\STATE For each explored menu $m$ in the previous phase, use $B_m(\gamma_1)$ samples to compute $\Delta_m$ stated in Lemma \ref{lem:deltam}.
			\STATE If there does not exist a state $w$ which is consistent with $\Delta_m$ ($\Delta_m = D_m(\omega)$) for all explored menu $m$, pick an arbitrary state $\omega$ and set $\Delta_m \leftarrow D_m(\omega)$ for all explored menu $m$. This step just make sure the number of signals is bounded by $|\varOmega|$.
			\STATE $l \leftarrow l + 1$.
			\STATE Set $S_l$ to be the collection of $\Delta_m$'s for all explored menu $m$.
		\ENDIF
	\ELSE
		\STATE \textbf{Exploitation:}
		\STATE If this is the first exploitation round, for each explored menu $m$ in the exploration, use $B_m(\gamma_2)$ samples to compute $\Delta_m$ stated in Lemma \ref{lem:deltam}. Set $S_l$ to be the collection of $\Delta_m$'s for all explored menu $m$.
		\STATE Suggest the menu which consists of the best action of each type conditioned on $S_l$ and the prior.
	\ENDIF
	\ENDFOR
     \end{algorithmic}
\end{algorithm}

First of all, it's easy to check by Claim \ref{clm:maxexplore_nocc} that for each agent, it is $\delta$-BIC to follow the recommended action if previous agents all follow the recommended actions. Therefore we have the following claim.
\begin{claim}
\label{clm:nocc_BIC}
Algorithm \ref{alg:nocc_main} is $\delta$-BIC.
\end{claim}


\begin{lemma}
\label{lem:exp_nocc}
For any $l > 0$, assume Algorithm \ref{alg:nocc_main} has at least $\min(l, |\M|)$ phases.
For a given state $\omega$, if a menu $m$ can be explored by a BIC recommendation policy $\pi$ at round $l$ (i.e. $ \Pr[\pi^l= m]> 0$), then such menu is guaranteed to be explored $B_m$ times by Algorithm \ref{alg:nocc_main} by the end of phase $\min(l, |\M|)$.
\end{lemma}

\begin{proof}
The proof is similar to Lemma \ref{lem:exp_public}. We prove by induction on $l$ for $l \leq |\M|$.

%Base case $l=1$ is trivial by Claim \ref{clm:maxexplore}. Assuming the lemma is correct for $l-1$, let's prove it's correct for $l$.

Let $S$ be the signal of Algorithm \ref{alg:nocc_main} in phase $l$. Let $S'$ be the history of $\pi$ in the first $l-1$ rounds. More precisely, $S' = R, H_1,...,H_{l-1}$. Here $R$ is the internal randomness of $\pi$ and $H_t = (M_t, A_t,u(\Theta_t, M_t(\Theta_t), \omega_0))$ is the menu and the action-reward pair in round $t$ of $\pi$.

Let $\M'$ to be the set of menus explored in the first $l-1$ phases of Algorithm \ref{alg:nocc_main}. By the induction hypothesis, we have $\forall t\in[l-1]$, $M_t \subseteq \M'$.

First of all, we have
\[
I(S'; \omega_0| S) = I(R,H_1,...,H_{l-1}; \omega_0| S)  = I(R; \omega_0| S) + I(H_1,...,H_{l-1}; \omega_0|S, R) = I(H_1,...,H_{l-1}; \omega_0|S, R).
\]

By the chain rule of mutual information, we have
\[
 I(H_1,...,H_{l-1}; \omega_0|S, R) = I(H_1;\omega_0|S,R) + I(H_2;\omega_0|S, R ,H_1) + \cdots + I(H_{l-1}; \omega_0|S,R,H_1,...,H_{l-2}).
\]

For all $t \in [l-1]$, we have
\begin{align*}
&I(H_t; \omega_0|S,R,H_1,...,H_{t-1}) \\
=& I(M_t,A_t, u(\Theta_t, M_t(\Theta_t), \omega_0); \omega_0|S,R,H_1,...,H_{t-1}) \\
=& I(A_t, u(\Theta_t, M_t(\Theta_t), \omega_0); \omega_0 | S,R,H_1,...,H_{t-1}, M_t)\\
\leq& I(D_{M_t}; \omega_0|S,R,H_1,...,H_{t-1},M_t). \\
\end{align*}
The second last step comes from the fact that $M_t$ is a deterministic function of $R,H_1,...,H_{t-1}$. The last step comes from the fact that $(A_t,u(\Theta_t, M_t(\Theta_t), \omega_0))$ is independent with $\omega_0$ given $D_{M_t}$.

Then we have
\begin{align*}
& I(D_{M_t}; \omega_0|S,R,H_1,...,H_{t-1},M_t)\\
=& \sum_{m \in \M'} \Pr[M_t = m] \cdot I(D_m;\omega_0 | S,R,H_1,...,H_{t-1},M_t = m)\\
\leq& \sum_{m \in M'} \Pr[M_t = m] \cdot I(D_m;\omega_0| \Delta_m, M_t =m).\\
\leq& \sum_{m \in M'} \Pr[M_t = m] \cdot H(D_m| \Delta_m, M_t =m).
\end{align*}
The last step comes from the fact that $I(D_m; (S\backslash \Delta_m),R,H_1,...,H_{t-1}|\omega_0, \Delta_m, M_t =m) = 0$. By Lemma \ref{lem:deltam}, we know that $\Pr[D_m \neq \Delta_m|M_t = m] \leq \gamma_1$. By Fano's inequality, we have
\[
H(D_m| \Delta_m, M_t =m) \leq H(\gamma_1) + \gamma_1 \log(|\varOmega| - 1) \leq 2\sqrt{\gamma_1} + \gamma_1 \log(|\varOmega| - 1)\leq \frac{\delta^2}{16|\M|}+\frac{\delta^2}{16|\M|}  = \frac{\delta^2}{8|\M|}.
\]

Therefore we have
\[
I(H_t; \omega_0|S,R,H_1,...,H_{t-1}) \leq \frac{\delta^2}{8|\M|}, \forall t \in [l-1].
\]
Then we get
\[
I(S'; \omega_0 | S) \leq \delta^2/8.
\]

By Lemma \ref{lem:ainfomono}, we know that $\EX_{s'}[\S'] \subseteq \EX^{\delta}_s[\S]$. By Claim \ref{clm:maxexplore_nocc}, we know that phase $l$ will explore menu $m$ at least $B_m(\gamma_0)$ times.

When $l > |\M|$, the proof follows from the same argument as the last paragraph of the proof of Lemma \ref{lem:exp_public}.
\end{proof}

\begin{corollary}[Restatement of Theorem \ref{thm:private_nocc}]
\label{cor:private_nocc}
For any $\delta > 0$, we have a $\delta$-BIC recommendation policy of $T$ rounds with expected total reward at least $\left(T - C\cdot \log(T) \right) \cdot \OPT$ for some constant $C$ which does not depend on $T$.
\end{corollary}

\begin{proof}

First of all, by Claim \ref{clm:nocc_BIC}, Algorithm \ref{alg:nocc_main} is $\delta$-BIC.

By Lemma \ref{lem:exp_nocc}, for each state $\omega$, Algorithm \ref{alg:nocc_main} explores all the eventually-explorable menus (i.e. $\MExp_{\omega}$) for by the end of $|\M|$ phases.

After that, by Lemma \ref{lem:deltam} and $\gamma_2 = \frac{1}{T|\M|}$, for a fixed state $\omega$, we know that with probability $1- 1/T$, $\delta_m = D_m$ for all $m \in \MExp_{\omega}$. In this case, the agent of type $\theta$ gets expected reward at least $u(\theta,m^*(\theta),\omega)$ where menu $m^* =\arg\max_{m \in \MExp_{\omega}} \sum_{\theta \in \varTheta} \Pr[\theta] \cdot u(\theta, m(\theta), \omega)$. Taking average over types, the expected reward per round should be at least $(1-1/T) \cdot \max_{m \in \MExp_{\omega}} \sum_{\theta \in \varTheta} \Pr[\theta] \cdot u(\theta, m(\theta), \omega)$.

We know that the expected number of rounds of the first $|\M|$ phases is $|\M| \cdot L = O(\ln(T))$. Therefore, Algorithm \ref{alg:nocc_main} has expected total reward at least $T \cdot \OPT- T \cdot (1/T) - O(\ln(T)) = T\cdot \OPT - O(\ln(T))$.

\end{proof}

